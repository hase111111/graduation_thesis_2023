%%%%%%%%%%%%%%%%%%%%%%%%%%%%%%%%%%%%%%%%%%%%%%%%%%%%%%%%%%%%%%%%%%%%%%%%
%%
%% 序論.tex
%% LaTeX-2e 専用
%% 
%% 
%%        設計工学研究室 学位論文テンプレート
%%
%%                      作成日時    2010年 12月 17日
%%
%%%%%%%%%%%%%%%%%%%%%%%%%%%%%%%%%%%%%%%%%%%%%%%%%%%%%%%%%%%%%%%%%%%%%%%%

\chapter{序論}\label{chapter:序論}
第\ref{chapter:序論}章では,本研究の研究背景と先行研究,そして研究の目的を述べる.


\section{研究背景}
近年,人間に代わって作業を行う移動ロボットの導入が進められている.
これらのロボットの多くはタイヤやホイールを用いての移動を行うが,その他の移動様式として,脚を使用して移動を行う多脚ロボットが存在する.
多脚ロボットは他の移動様式を用いて移動するロボットに比べて,障害物をまたいで超えることが可能な点や,離散的に接地点を選択できる点において優れているといえる.

実際に,林業を行う山間地において多脚ロボットの導入を

このような不整地において,多脚ロボットを使用する場合は適切な歩容計画を行う必要がある.
歩容計画には,カムやリンクを用いて,周期的に脚を動かす固定歩容と,
非周期的に脚を動かす自由歩容がある.

当研究室で行われてきた先行研究では,
\section{本研究の目的}


\section{本論文の構成}
本論文は,全x章から構成される.

第2章「理論と実施計画」では,~を述べる.
第3章「実験装置や開発機械」では,~を述べる.
第4章「実験」では,~を述べる.
第5章「結論」では本論文の結論と今後の課題を述べる.

\section{other}
図の参照はFig. \ref{fig:sample}とする.

表の参照はTab. \ref{table:sample}とする.

参考文献の参照は\cite{実用的4足歩行機械}とする.



%図の挿入例
\begin{figure}[tbp]
  \begin{center}
    \includegraphics[width=50mm, clip]{figure1/sample.eps}
    \caption{Sample figure}
    \label{fig:sample}
  \end{center}
\end{figure}

%表の挿入例
\begin{table}[tbp]
    \caption{Sample table}
    \label{table:sample}
    \begin{center}
        \begin{tabular} {|c|c|c|}
        \hline
        Item & Spec. & Quantity  \\
        \hline\hline
        A & high & 10 \\
        \hline
        B & low & 100 \\
        \hline
        \end{tabular}
    \end{center}
\end{table}
