%%%%%%%%%%%%%%%%%%%%%%%%%%%%%%%%%%%%%%%%%%%%%%%%%%%%%%%%%%%%%%%%%%%%%%%%
%%
%% 結論.tex
%% LaTeX-2e 専用
%% 
%% 
%%        設計工学研究室 学位論文テンプレート
%%
%%                      作成日時    2010年 12月 17日
%%
%%%%%%%%%%%%%%%%%%%%%%%%%%%%%%%%%%%%%%%%%%%%%%%%%%%%%%%%%%%%%%%%%%%%%%%%

\chapter{常に脚軌道生成が可能な自由歩容パターン生成手法を用いた実機実験}\label{chapter:常に脚軌道生成が可能な自由歩容パターン生成手法を用いた実機実験}

\chref{chapter:常に脚軌道生成が可能な自由歩容パターン生成手法を用いた実機実験}では,
実機を用いた歩行実験を行い,

\section{実験目的}

本論文では,~~を論じた.

第1章「序論」では,~を述べた.
第2章「理論と実施計画」では,~を述べた.
第3章「実験装置や開発機械」では,~を述べた.
第4章「実験」では,~を述べた.
第5章「結論」では本論文の結論と今後の課題を述べた.

\section{実験に使用した6脚ロボット}


\section{歩行条件}

\section{実験に使用した地形}

\section{結果}

\section{考察}