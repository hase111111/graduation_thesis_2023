
\chapter{常に脚軌道生成が可能な自由歩容パターン生成手法を用いた実機実験}\label{chapter:常に脚軌道生成が可能な自由歩容パターン生成手法を用いた実機実験}

\chref{chapter:常に脚軌道生成が可能な自由歩容パターン生成手法を用いた実機実験}では,
実機を用いた歩行実験を行い,
本研究で提案した自由歩容パターン生成手法によって脚軌道生成の失敗が生じないことを示す.

\section{実験目的}
これまで,シミュレーションを用いた歩行実験を行ってきたが,
本研究で用いたシミュレーション環境では脚先の滑りによるずれやモータのトルクを考慮してない.
そのため,実機を用いた歩行実験を行い,
本研究で提案した自由歩容パターン生成手法によって脚軌道生成の失敗が生じないことを示す.

\section{実験に使用した6脚ロボット}


\section{歩行条件}

\section{実験に使用した地形}
実験に用いた地形は

\section{結果}

\section{考察}