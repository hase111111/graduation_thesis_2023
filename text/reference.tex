\begin{thebibliography}{99}
    \bibitem{Sotnik_Prospects_for_Introduction}
    Sotnik S, Lyashenko V:
    ``Prospects for Introduction of Robotics in Service'',
    International Journal of Academic Engineering Research.
    Vol. 6,
    pp.4-9,
    2022.

    \bibitem{Pudu_BellaBot}
    Pudu Robotics Inc:
    ``BellaBot'',
    https://www.pudurobotics.com/jp/products/bellabot (参照2024/01/23).

    \bibitem{Locomotion_for_difficult_terrain}
    Freyr Hardarson:
    ``Locomotion for difficult terrain'',
    1997.

    \bibitem{Boston_Dynamics_Spot}
    Boston Dynamics Inc:
    ``Spot®'',
    https://bostondynamics.com/products/spot/ (参照2024/01/23).

    \bibitem{NEDO}
    国立研究開発法人 新エネルギー・産業技術総合開発機構:
    ``NEDO 先導研究プログラム 2021年度'',
    Vol.1,
    p.57, 
    2022. 

    \bibitem{J_Kim_Dexterous_Crabster}
    B. H. Jun, Hyungwon Shim:
    ``A Dexterous Crabster Robot Explores the Seafloor'',
    The ACM Magazine for Students,
    Vol.20,
    pp.38-45,
    2014.

    \bibitem{J_Kim_Little_Crabster}
    J. Y. Kim, B. H. Jun:
    ``Mechanical Design of Six-Legged Walking Robot, Little Crabster'',
    Oceans - Yeosu,
    pp.1-8,
    2012.

    \bibitem{Hirose_Static_stability_criterion}
    広瀬,塚越,米田: 
    ``不整地における歩行機械の静的安定性評価基準'', 
    J. of Robotic Systems,
    Vol.16, No.8, 
    pp.1076-1082, 
    1998.

    \bibitem{Prabir_Graph_search}
    Prabir K. Pal, K. Jayarajan: 
    ``Generation of Free Gait A Graph Search Approach'',
    IEEE Transactions on Robotics and Automation,
    Vol.7, No.3,
    1991.

    \bibitem{Prabir_Graph_search_Six}
    Prabir K. Pal, V. Mahadev and K. Jayarajan:
    ``Gait generation for a six-legged walking machine through graph search'',
    Proceedings of the 1994 IEEE International Conference on Robotics and Automation,
    vol.2,
    pp.1332-1337,
    1994.

    \bibitem{Oki_Graph_search}
    大木,程嶋,琴坂: 
    ``多脚ロボットの不整地踏破を目標とするグラフ探索を用いた歩行パターン生成'', 
    ロボティクス・メカトロニクス講演会講演概要集,
    2015.   

    \bibitem{Nakaoka_Graph_search}
    中岡,程嶋,琴坂: 
    ``不整地における特定位置・脚着地点への遷移を目的とした多脚歩行ロボットの歩行動作計画'',
    日本機械学会関東支部総会講演会講演論文集,
    2016.

    \bibitem{Shina_Graph_search}
    椎名,程嶋,琴坂: 
    ``グラフ探索を用いた多脚ロボットの旋回歩容パターン生成'',
    日本機械学会関東支部総会講演会講演論文集,
    2018.

    \bibitem{Miura_Graph_search}
    三浦,程嶋,琴坂: 
    ``グラフ探索による多脚歩行ロボットの自由歩容パターン生成 第4報:出現頻度によるノード枝刈りを用いた探索時間の短縮'',
    日本機械学会関東支部総会講演会講演論文集,
    2019.

    \bibitem{Hato_Graph_search}
    波東,程嶋,琴坂: 
    ``グラフ探索による多脚歩行ロボットの自由歩容パターン生成 第5報: 重心の上下移動のエッジを用いた重心高さの変更'',
    日本機械学会関東支部総会講演会講演論文集,
    2020.

    \bibitem{Thomas_C++20}
    Thomas Koppe:
    ``Changes between C++17 and C++20 DIS''.
    ISO/IEC JTC1 SC22 WG21 P2131R0.
    2020.
    https://www.open-std.org/jtc1/sc22/wg21/docs/papers/2020/p2131r0.html, 
    (参照2024/01/23).


\end{thebibliography}
\endinput