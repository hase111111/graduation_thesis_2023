

\chapter{結論}\label{chapter:結論}

\section{結論}

本論文では,まずグラフ探索による自由歩容パターン生成手法の先行研究の問題点を指摘し,
その問題点を解決するための歩容パターンの再評価手法を提案した.
次に再評価手法の実装方法を述べ,実装した再評価手法を用いてシミュレーション実験を行いその結果を示した.
そして,シミュレーションの結果より本問題において再評価手法は有効でないことと,
適切なロボットのモデル化によって先行研究の問題点を解決できることを示した.

第1章「序論」では,グラフ探索による自由歩容パターン生成手法の利点を示し,
同時に先行研究で確認された脚軌道生成の失敗を述べた.

第2章「歩容パターンの再評価手法の提案」では,
シミュレーション実験によって脚軌道生成の失敗を確認し,
脚軌道生成失敗を防ぐ手段として再評価手法を提案した.

第3章「歩容パターンの再評価手法の実装」では,
2章で述べた再評価手法の実装方法を述べた.

第4章「再評価手法の有効性の確認のための歩行シミュレーション」では,
実装した再評価手法を用いたシミュレーション実験の結果を示し,
再評価手法が有効でないことを示した.

第5章「常に脚軌道生成が可能な自由歩容パターン生成手法を用いた実機実験」では

TODO:実機試験終了後に記述する.

\section{今後の展望}
本研究が取り扱った問題については,再評価手法は適切な手法ではないことが示されたが,
他の問題については再評価手法は有効である可能性がある.
具体的にはトルク不足による脚の沈み込みである.
グラフ探索の処理では計算時間を短縮するために,
力学的な計算を行っていない.
しかしそれを原因として,実機実験時にトルク不足による脚の沈み込みが生じてしまうことが確認されている.
そこで,グラフ探索によって得られた歩容パターンを実機で実行する前に,
力学的な計算を用いてトルクが不足していないかを確認し,
トルク不足が生じている場合は再評価手法を用いて歩容パターンを再評価することで,
トルク不足による脚の沈み込みを防ぐことができると考えられる.

