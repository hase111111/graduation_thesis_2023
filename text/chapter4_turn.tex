
\subsection{シミュレーション実験の目的}
直進動作のシミュレーション実験によって,脚軌道生成の失敗を防ぐためには,最小半径を140mmとすることが有効であるとわかった.
しかし,最小半径を140mmに設定すると,近似された脚の可動範囲が小さくなる.
そのため,先行研究の手法で歩行することができた地形であっても,歩行することができなくなる可能性がある.
直進動作時については,歩行することができることが確認できたため,
本章では,旋回動作を行う際に,最小半径を140mmとした場合に,歩行することができるかを検証することを目的とする.

先行研究では旋回動作は2次元空間でのみ実装されていたが,新たに3次元空間での旋回動作を実装したため,
先行研究で確認された地形を含めた3次元空間での旋回動作をシミュレーションで検証する.

\subsection{シミュレーション実験の条件}