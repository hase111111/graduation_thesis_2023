
\subsection{シミュレーション実験の目的}
4.1章,4.2章のシミュレーション実験によって,
最小半径を140mmに設定することで,
脚軌道生成に失敗することなく直進動作,旋回動作を行えることが確認された.
そこで,これらの直進動作と旋回動作を組み合わせた場合でも,
最小半径を140mmに設定することで,脚軌道生成に失敗することなく
歩行することができるかを検証することを目的とする.

グラフ探索による自由歩容パターン生成手法を統合したプログラムを用いて,
直進動作と旋回動作を組み合わせた歩容パターンを生成し,
脚軌道生成に失敗することなく歩行できること,
また指定した通りに動作を切り替えることができることを確認する.

\subsection{シミュレーション実験の条件}
シミュレーションを行うソフトウェア,シミュレーションを行う計算環境,
モデルとするロボットは直進動作のシミュレーション実験と同様であるため説明を省略する.
歩行条件についても直進動作のシミュレーション実験および,
旋回動作のシミュレーション実験と同様とする.

\subsubsection{シミュレーションの手順}


\subsection{シミュレーション実験の結果}

\subsection{考察}
