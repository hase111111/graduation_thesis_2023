
\chapter{序論}\label{chapter:序論}
第\ref{chapter:序論}章では,本研究の背景と先行研究,そして研究の目的を述べる.


\section{背景}
\subsection{多脚ロボットの特徴}
\subsubsection{不整地における多脚ロボット}
近年,人間に代わって作業を行う移動ロボットの導入が進められている.
これらのロボットの多くはタイヤやクローラを用いての移動を行うが,
その他の移動様式として,脚を使用して移動を行う脚ロボットが存在する.
脚ロボットは他の移動様式を用いて移動するロボットに比べて,
障害物をまたいで超えることが可能な点や,離散的に接地点を選択できる特徴を持っている\cite{Locomotion_for_difficult_terrain}.
そのため,タイヤでは移動できないような不整地においても移動することが可能であり,
クローラではスリップしてしまうような環境においても移動することが可能である.
実際に,林業を行う山間地において脚ロボットを導入し,
作業を行う実証実験が行われている\cite{NEDO}.
また,離散的に接接地地点を選択できる特徴を活かし,海底で作業を行うロボットとしても研究が行われている.
これは,クローラによる移動では海底の砂が巻きあがってしまい,カメラやセンサを遮ってしまうためである.
しかし,他の移動様式と比較して,脚ロボットは歩行制御が難しくなるという問題がある.

\subsubsection{脚数による多脚ロボットの分別}
前述したような特性を持つ多脚ロボットであるが,その性能は脚数によって変化する.
ロボットの性能の指標として,歩行速度や消費エネルギーなどがあるが,不整地において用いることを考え,
静的安定余裕\cite{Hirose_Static_stability_criterion}(Stability Margin)を指標として分別することとする.
静的安定余裕とは,ロボットが静的に安定するために必要な脚位置と重心位置の関係を表す指標である.
支持脚を結んでできる多角形の重心が,支持脚の内部にある場合,ロボットは静的に安定する.
そのため,静的安定余裕では多角形の辺から重心までの距離を評価に用いる.
\begin{description}
  \item[2脚ロボット]\mbox{}\\
    2脚ロボットは,人間のように歩行を行うことができる.
    しかし,安定性を確保するためには,歩行速度を遅くする必要がある.
    そのため,2脚ロボットは歩行速度が遅く,消費エネルギーが大きいという特徴を持つ.
  \item[4脚ロボット]
  \item[6脚ロボット]
  \item[脚数が6脚よりも多いロボット] 
\end{description}

\subsection{固定歩容と自由歩容}
多脚ロボットが歩行を行う際には,脚を適切な順番で動かす必要がある.
多脚ロボットにおいて,脚の動かし方を歩容と呼ぶ.
歩容にはさまざまな種類があるが,大きく分別するとカムやリンクを用いて,
周期的に脚を動かす固定歩容と,非周期的に脚を動かす自由歩容がある.

\subsection{グラフ探索による自由歩容パターン生成手法}
本研究室においては,6脚ロボットの自由歩容パターン生成手法として,グラフ探索による由歩容パターン生成手法を提案してきた.
グラフ探索による自由歩容パターン生成手法は,4脚ロボットにおいて行われていたが,6脚ロボットにおいては行われていなかった.
これは,脚の本数が増えることで,脚の動かし方の組み合わせが増えるため,実時間内の計算が困難になるためである.

そこで,本研究室では,グラフ探索による自由歩容パターン生成手法を6脚ロボットに適用するため,
脚位置の離散化を行うことによるグラフの階層構造化を行った.また,グラフ探索による自由歩容パターン生成と脚軌道生成を分離することで,
グラフ探索による自由歩容パターン生成手法を6脚ロボットに適用することに成功した.

\section{本研究の目的}
これまでの研究によって,3次元の不整地において,重心高さを変更しつつ,
自由歩容パターン生成を行うことが可能となった.
しかし低頻度ではあるが,グラフ探索に成功したとしても脚軌道が生成できず,その歩容パターン通りに歩行することできなくなり,
動作を停止してしまう問題が生じてしまった.

そこで本論文では,常に脚軌道生成に成功するような歩容パターン生成手法を提案し,
脚軌道生成の失敗による動作停止を防ぐことを目的とする.

\section{本論文の構成}
本論文は,全6章から構成される.

第2章「歩容パターンの再評価手法の提案」では,常に脚軌道生成が可能になる手法として,
歩容パターンの再評価手法を提案し,その機能を述べる.

第3章「歩容パターンの再評価手法の実装」では,提案したプログラムの実装方法を述べる.

第4章「再評価手法の有効性の確認のための歩行シミュレーション」では,
提案手法を用いたシミュレーション実験の結果を述べる.

第5章「常に脚軌道生成が可能な自由歩容パターン生成手法を用いた実機による歩行実験」
では,提案手法を用いた実機試験の結果を述べる.

第6章「結論」では本論文の結論と今後の課題を述べる.
