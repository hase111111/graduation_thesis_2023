%% 歩容パターンの再評価手法の提案.tex
%% LaTeX-2e 専用

\chapter{歩容パターンの再評価手法の提案}\label{chapter:歩容パターンの再評価手法の提案}
第\ref{chapter:歩容パターンの再評価手法の提案}章では,先行研究の問題点を指摘し,
常に脚軌道生成が可能な自由歩容パターン生成手法として,歩容パターンの再評価手法を述べる.

\section{本研究室における自由歩容パターン生成の先行研究}


\section{歩行シミュレーションによる脚軌道生成失敗時の脚先位置の特定}

\subsection{シミュレーション実験の目的}
脚軌道生成の失敗の失敗を防ぐためには,脚軌道生成の失敗時の脚先の座標を特定する必要がある.
そのため,予備実験として先行研究と同じ条件で歩行シミュレーション実験を行い,失敗の原因を考察した.

\subsection{シミュレーションの条件}

\subsection{シミュレーションの結果}
以下の図に脚軌道生成失敗時の脚先の座標を示す.

\subsection{脚軌道生成の失敗の原因の考察}

\section{常に脚軌道生成が可能な自由歩容パターン生成手法の検討}
常に脚軌道生成を可能にするためには,近似された脚可動範囲を適切に設定する必要がある.

\section{歩容パターンの再評価手法}

