%%%%%%%%%%%%%%%%%%%%%%%%%%%%%%%%%%%%%%%%%%%%%%%%%%%%%%%%%%%%%%%%%%%%%%%%
%%
%% 実験装置や開発機械.tex
%% LaTeX-2e 専用
%% 
%% 
%%        設計工学研究室 学位論文テンプレート
%%
%%                      作成日時    2010年 12月 17日
%%
%%%%%%%%%%%%%%%%%%%%%%%%%%%%%%%%%%%%%%%%%%%%%%%%%%%%%%%%%%%%%%%%%%%%%%%%

\chapter{歩容パターンの再評価手法の実装}\label{chapter:歩容パターンの再評価手法の実装}
第\ref{chapter:歩容パターンの再評価手法の実装}章では,
第\ref{chapter:歩容パターンの再評価手法の提案}章で述べた歩容パターンの再評価手法の実装方法について述べる.

\section{グラフ探索による自由歩容パターン生成手法の実装}
\section{歩容パターンの再評価手法の実装}
\section{グラフ探索による自由歩容パターン生成手法の統合}
先行研究において,グラフ探索による自由歩容パターン生成手法はロボットの動作によって別のものを使用しており,
それぞれ別のプログラムで実装されている.
不整地の踏破を行うためには,さまざまな動作を組み合わせて使用する必要がある.
そのため本研究では,グラフ探索による自由歩容パターン生成手法を統合し,
1つのプログラムで実行できるようにした.

\subsection{ロボットの動作}
先行研究において,すでに実装されているロボットの動作を表\ref{tab:ロボットの動作}に示す.
表において,2次元空間とはロボットが平面上で動作することを表し,
3次元空間とはロボットが立体的な地形で動作することを表す.

\begin{table}[htbp]
    \caption{実装済みのロボットの動作}
    \label{tab:ロボットの動作}
    \begin{center}
        \begin{tabular}{|c|c|c|c|c|c|c|} \hline
            \backslashbox{動作}{ロボット} & 2次元空間 & 3次元空間  \\ \hline
            直進 & $\bigcirc$ & $\bigcirc$ \\ \hline
            その場旋回 & $\bigcirc$ & $\times$ \\ \hline
            旋回 & $\bigcirc$ & $\times$ \\ \hline
            旋回 & $\bigcirc$ & $\times$ \\ \hline
            特定姿勢での静止 & $\bigcirc$ & $\times$ \\ \hline
            %% まるはtexにおいて,$\bigcirc$
            %% ばつはtexにおいて,$\times$
        \end{tabular}
    \end{center}
\end{table}

\subsection{自由歩容パターン生成手法の切り替えアルゴリズム}
