%% 歩容パターンの再評価手法の実装.tex
%% LaTeX-2e 専用

\chapter{歩容パターンの再評価手法の実装}\label{chapter:歩容パターンの再評価手法の実装}
第\ref{chapter:歩容パターンの再評価手法の実装}章では,
第\ref{chapter:歩容パターンの再評価手法の提案}章で述べた歩容パターンの再評価手法の実装方法について述べる.

\section{グラフ探索による自由歩容パターン生成手法の実装}
再評価手法の実装方法の説明の前に,グラフ探索による自由歩容パターン生成手法の実装方法について述べる.

\subsection{ノードの定義}
まず,歩容パターングラフのノードについて説明する.
歩容パターングラフのノードは\figref{fig:robot_state_node}のような,
ロボットの状態を表す構造体RobotStateNodeで表現される.
なお,\figref{fig:robot_state_node}はUML(Unified~Modeling~Language)を用いて表現されており,
上からクラス・構造体の名前,メンバ変数,メンバ関数を表している.
メンバ変数は,アクセッサ,変数名,型の順に記述されており,
メンバ関数は,アクセッサ,関数名,引数,戻り値の順に記述されている.
また,先頭に付いている+はpublicを表し,protectedは\#,
privateは$-$で表す.

RobotStateNodeのメンバ変数leg\_stateは,ロボットの脚の状態を表すビット列である.

RobotStateNode構造体とVector3構造体は
それぞれメンバ変数を操作するためのメンバ関数を持っているが,
グラフ探索とは関係がないため,ここでは説明を省略する.

\begin{figure}[htbp]
  \centering
  \begin{tikzpicture}
    \begin{class}[text width=8cm]{RobotStateNode}{0, 0}
      \attribute{+ leg\_state : std::bitset\textless 28\textgreater}
      \attribute{+ leg\_pos : std::array\textless Vector3, 6\textgreater}
      \attribute{+ leg\_reference\_pos : std::array\textless Vector3, 6\textgreater}
      \attribute{+ center\_of\_mass\_global\_coord : Vector3}
      \attribute{+ posture : Quaternion}
      \attribute{+ next\_move : HexapodMove}
      \attribute{+ parent\_index : int}
      \attribute{+ depth : int} 
      \operation{}
    \end{class}
  \end{tikzpicture}
  \caption{RobotStateNode Struct}
  \label{fig:robot_state_node}  % chktex 24
\end{figure}

\begin{figure}[htbp]
  \centering
  \begin{tikzpicture}
    \begin{class}[text width=8cm]{Vector3}{0, 0}
      \attribute{+ x : float}
      \attribute{+ y : float}
      \attribute{+ z : float}
      \operation{}
    \end{class}
  \end{tikzpicture}
  \caption{Vector3 Struct}
  \label{fig:Vector3}  % chktex 24
\end{figure}

\section{歩容パターンの再評価手法の実装}

\section{グラフ探索による自由歩容パターン生成手法の統合}
先行研究において,グラフ探索による自由歩容パターン生成手法はロボットの動作によって別のものを使用しており,
それぞれ別のプログラムで実装されている.
不整地の踏破を行うためには,さまざまな動作を組み合わせて使用する必要がある.
そのため本研究では,グラフ探索による自由歩容パターン生成手法を統合し,
1つのプログラムで実行できるようにした.

\subsection{ロボットの動作}
先行研究において,すでに実装されているロボットの動作を表\ref{tab:ロボットの動作}に示す.
表において,2次元空間とはロボットが平面上で動作することを表し,
3次元空間とはロボットが立体的な地形で動作することを表す.

\begin{table}[htbp]
	\caption{実装済みのロボットの動作}
	\label{tab:ロボットの動作}  % chktex 24
	\begin{center}
   	\begin{tabular}{|c|c|c|c|c|c|c|} \hline  % chktex 44
    	\backslashbox{動作}{ロボット} & 2次元空間 & 3次元空間  \\ \hline  % chktex 44
      直進 & $\bigcirc$ & $\bigcirc$ \\ \hline  % chktex 44
      その場旋回 & $\bigcirc$ & $\times$ \\ \hline  % chktex 44
      旋回 & $\bigcirc$ & $\times$ \\ \hline  % chktex 44
      旋回 & $\bigcirc$ & $\times$ \\ \hline  % chktex 44
      特定姿勢での静止 & $\bigcirc$ & $\times$ \\ \hline  % chktex 44
      %% まるはtexにおいて,$\bigcirc$
      %% ばつはtexにおいて,$\times$
    \end{tabular}
  \end{center}
\end{table}

\subsection{自由歩容パターン生成手法の切り替えアルゴリズム}
