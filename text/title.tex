%%%%%%%%%%%%%%%%%%%%%%%%%%%%%%%%%%%%%%%%%%%%%%%%%%%%%%%%%%%%%%%%%%%%%%%%
%%
%% 題目.tex
%% LaTeX-2e 専用
%% 
%% 
%%        設計工学研究室 学位論文テンプレート
%%
%%                      作成日時    2020年 2月 11日
%%
%%%%%%%%%%%%%%%%%%%%%%%%%%%%%%%%%%%%%%%%%%%%%%%%%%%%%%%%%%%%%%%%%%%%%%%%
\thispagestyle{empty}
\newcommand{\ctext}[1]{\textcolor[rgb]{0.65, 0.65, 0.65}{\raise0.2ex\hbox{\textcircled{\scriptsize{#1}}}}}
%% 卒業論文題目     
\begin{center}
        {\huge 埼玉大学 工学部} \\
        {\huge 機械工学科} \\        
        \vspace{10mm}
        {\Huge 令和5年度 \quad 卒業論文}\\
        \vspace{10mm} 
        
        {\Huge グラフ探索を用いた多脚ロボットの歩容パターン生成における脚軌道生成失敗時の歩容パターンの再評価手法} \\
        \vspace{10mm}
        {\LARGE A Reevaluation Method of Gait Pattern Generation for Multilegged Robots Using Graph Search when Leg Trajectory Generation Fails} \\        
        \vspace{10mm} %題目の長さに応じて適宜修正すること
 \end{center}
        
 \begin{table}[!h]
        \begin{flushright}        
        \renewcommand{\arraystretch}{1.5}
        \begin{tabular}{|c|cc|}
            \hline
            {\Large 学科長}    & {\Large  荒居善雄 教授 \quad} & {\Large\ctext{印}}\\        
            \hline
            {\Large 主指導教員} & {\Large  琴坂信哉 准教授 } &{\Large\ctext{印}}\\
            \hline
            {\Large 副指導教員} & {\Large  程島竜一 准教授 }& \\
            \hline
        \end{tabular}
        \renewcommand{\arraystretch}{1.0}        
    \end{flushright}        
 \end{table}
\begin{table}[!b]
        \begin{flushright}
        \renewcommand{\arraystretch}{1.2}
        \begin{tabular}{|c|c|}
            \hline
             {\Large 提出日} & {\Large 2023年2月XX日}\\ %年は西暦で記載
             \hline
             {\Large 研究室} & {\Large 設計工学 }\\
             \hline
             {\Large 学籍番号} & {\Large 20TM028}\\
             \hline
             {\Large 氏  名} & {\Large 長谷川 大晴}\\
            \hline
        \end{tabular}
        \renewcommand{\arraystretch}{1.0}
        \end{flushright}
\end{table}
