

\chapter{再評価手法の有効性の確認のための歩行シミュレーション}\label{chapter:再評価手法の有効性の確認のための歩行シミュレーション}
第\ref{chapter:再評価手法の有効性の確認のための歩行シミュレーション}章では,
実装した再評価手法の有効性を確認するために,シミュレーションを用いた歩行実験を行い,その結果を示す.

\section{直進動作の自由歩容パターン生成シミュレーション}
まずは,直進動作の自由歩容パターン生成シミュレーションを行い,再評価手法の有効性を確認する.

\subsection{実験目的}
再評価手法によって生成された自由歩容パターンを用いて,脚軌道生成の失敗が生じないことを確認することをシミュレーション実験の目的とする.
波東らの研究で使用されていた地形を歩行させ,脚軌道生成の失敗の回数が0回となることを確認する.

\subsection{実験条件}
シミュレーションには,\chref{chapter:歩容パターンの再評価手法の提案}で使用したシミュレーション環境と同じ環境を使用する.
また,計算環境とモデルとするロボットも,\chref{chapter:歩容パターンの再評価手法の提案}と同じものを使用する.

実験に使用した地形は,波東らの研究で使用されていた地形を用いる.

\section{旋回動作の自由歩容パターン生成シミュレーション}

\section{動作統合時の自由歩容パターン生成シミュレーション}