
% 脚の可動班を表示するプログラムの説明

\chapter{脚の可動域を表示するプログラム}\label{chapter:leg_range_python}

\section{概要}
近似的な脚の可動域が適切であるかを評価するためには,
PhantomXの可動域を正確に把握する必要がある.
そのためには可動域の可視化を行うことが必要だろう.
そこで,PhantomXの脚の可動域を可視化するプログラムを作成した.
プログラムは簡単のためPythonを用いて作成しており,
GitHubを通じてだれでも利用できるようにしている.
以下にプログラムの導入方法を説明し,
プログラムの仕様について述べる.

\section{導入方法}
プログラムはGitHubを通じて公開しているため,
まずはGitHubからプログラムをダウンロードする方法を説明する.
そしてPythonのプログラムをコンパイルする方法を説明する.

\subsection{Gitの使用方法}
Gitとは,ソースコードなどの変更履歴を記録・追跡するための分散型バージョン管理システムである.
分かりやすい例を挙げるならば,Googleドキュメントの履歴機能や,
Microsoft Wordの変更履歴機能のようなシステムをさまざまなプログラムのソースコードに対して適用したものといえるだろう.
また,GitHubとは,Gitを用いてソースコードを管理するためのサービスである.
GitHubを用いることで,ソースコードの変更履歴を保存することができるだけでなく,
自分のソースコードを公開したり,他の人が作成したソースコードをダウンロードしたりすることもできる.

\subsection{pythonのコンパイル方法}
本項目では,Windows 10におけるpythonのコンパイル方法を説明する.
Mac OSやLinuxにおけるpythonのコンパイル方法については,ここでは説明しない.
(しかし,「python Mac 環境構築」,「python Linux 環境構築」などで検索すると最適な方法が見つかるだろう.)

\section{プログラムの仕様}

