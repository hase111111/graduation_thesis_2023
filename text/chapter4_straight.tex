まずは,直進動作の自由歩容パターン生成シミュレーションを行い,再評価手法の有効性を確認する.

\subsection{シミュレーション実験の目的}
シミュレーション実験の目的は,
再評価手法によって生成された自由歩容パターンを用いて,
脚軌道生成の失敗が生じないことを確認することとする.

波東らの研究で実機では歩行することができなかった地形を含む5種類の地形を歩行させ,
脚軌道生成の失敗の回数が0回となることを確認する.
また再評価手法の利点として,計算時間が大幅に伸びないことが期待されるため,計算にかかる時間も測定する.

\subsection{シミュレーション実験の条件}
シミュレーションには,\chref{chapter:歩容パターンの再評価手法の提案}で使用したシミュレーションソフトウェアを使用した.
また,計算環境も同じく\tableref{tab:simulation_env}に示したものを用いており,
シミュレーションのモデルとしたロボットも同じくPhantomXとした.

\subsubsection{歩行する地形}
シミュレーションに使用する地形を\figref{fig:ch5_simu_terrain}に示す.
また矢印でロボットの進行方向を示した.
シミュレーションに使用した地形は,平面,130mmの上り段差,130mmの下り段差,15度の上り斜面,15度の下り斜面の5種類である.
\chref{chapter:歩容パターンの再評価手法の提案}で行ったシミュレーションの条件から変更し,上り段差と下り段差の高さを130mmとしている.

\subsubsection{歩行する時の条件}
シミュレーション時の歩行条件は以下の通りである.
\begin{itemize}
  \item 胴体姿勢は常に水平である.
  \item 最小半径を$130 [mm]$とし,再評価時には最小半径を$140 [mm]$に変更する.
  \item 重心から見た遊脚高さを$-20 [mm]$とする.
  \item 胴体を地形から最小$30 [mm]$離す.
  \item 常に静的安定余裕は$15 [mm]$以上を保つ.
\end{itemize}
最小半径を\figref{fig:ch5_range_revaluation}に示す.
この図は\figref{fig:simu_leg_range}と同様に,横軸をx軸,縦軸をz軸としており,単位は$[mm]$である.
実際の可動範囲を黒の線で示し,近似された可動範囲を緑の線で示している.
また,再評価時の近似された可動範囲を黄緑の線で示しており,遊脚高さを赤い線で示している.

\begin{figure}[htbp]
  \centering
  \includegraphics[width=0.5\linewidth,trim={30 30 30 30}, clip]{figure/chapter4/revaluation_view.png}
  \caption{Leg Range of Motion for Revaluation Method}
  \label{fig:ch5_range_revaluation} % chktex 24
\end{figure}

\subsubsection{シミュレーションの手順}
各地形で水平方向に$1200 [mm]$直進するまでの自由歩容パターンを生成した.
\figref{fig:ch5_simu_terrain}のロボットの初期位置をランダムに変化させて計5回ずつ歩行させた.
再評価手法を用いた場合と用いていない場合を比較するため,
再評価手法を用いず初めから最小半径を$140 [mm]$としたプログラムを用いて同様のシミュレーションを行った.
シミュレーションごとに,脚軌道生成の失敗の回数,脚先座標,計算時間を測定した.
\\

% 地形の図
\begin{figure}[htbp]
  \begin{tabular}{cc}
    \begin{minipage}[t]{0.45\hsize}
      \begin{center}
      \includegraphics[width=1.0\linewidth]{figure/chapter4/map_flat.png}
      \text{(a) flat}
      \label{fig:ch5_simu_terrain_flat} % chktex 24
      \end{center}
    \end{minipage} 
    &
    \begin{minipage}[t]{0.45\hsize}
      \begin{center}
      \includegraphics[width=1.0\linewidth]{figure/chapter4/map_130mm.png}
      \text{(b) up step}
      \label{fig:ch5_simu_terrain_up_step} % chktex 24
      \end{center}  
    \end{minipage}
    \\
    &\\  % 空白を入れる
    \begin{minipage}[t]{0.45\hsize}
      \centering
      \includegraphics[width=1.0\linewidth]{figure/chapter4/map_-130mm.png}
      \centering
      \text{(c) down step}
      \label{fig:ch5_simu_terrain_down_step} % chktex 24
    \end{minipage} 
    &
    \begin{minipage}[t]{0.45\hsize}
      \centering
      \includegraphics[width=1.0\linewidth]{figure/chapter4/map_15deg.png}
      \centering
      \text{(d) up slope}
      \label{fig:up_slope_terrain} % chktex 24
    \end{minipage}    
    \\
    &\\  % 空白を入れる
    \begin{minipage}[t]{0.45\hsize}
      \centering
      \includegraphics[width=1.0\linewidth]{figure/chapter4/map_-15deg.png}
      \centering
      \text{(e) down slope}
      \label{fig:down_slope_terrain} % chktex 24
    \end{minipage}     
    &
    \\
  \end{tabular}
  \caption{Terrain}
  \label{fig:ch5_simu_terrain} % chktex 24
\end{figure}

\newpage

\subsection{シミュレーション実験の結果}
\subsubsection{脚軌道生成の失敗の回数}
再評価手法を用いた場合の脚軌道生成の失敗の回数を\tableref{tab:ch5_failure_count_1}に示す.
また,再評価手法を用いない場合の脚軌道生成の失敗の回数を\tableref{tab:ch5_failure_count_2}に示す.
この結果から,再評価手法を用いた場合でも,脚軌道生成の失敗が生じてしまうことがわかる.
失敗する場合は,遊脚時に脚軌道が可動範囲外を通ることがわかる.
加えて,初めから最小半径を140mmとすれば,脚軌道生成の失敗は生じていないことがわかる.

\subsubsection{脚先座標}
再評価手法を用いた場合の,脚先座標を地形ごとに\figref{fig:ch5_simu_res_1}に示した.
各図は\figref{fig:ch5_range_revaluation}の$50<x<250,-200<z<0$の範囲を拡大したものである.
脚先座標は支持脚時を青い丸点,遊脚時を赤い丸点で示している.
また,脚軌道生成に失敗した際の脚軌道の中継点を水色の$\times$で示している.
同様に再評価手法を用いない場合の脚先座標を\figref{fig:ch5_simu_res_2}に示した.

これらの結果から,再評価手法を用いた場合でも脚軌道が,
$130 < x < 140$,$-30 < z < -20$の脚の可動範囲外となる領域を通ることがわかる.
また,最初から最小半径を$140 [mm]$とした場合は,脚軌道が可動範囲外を通ることはないことがわかる.

\subsubsection{計算時間}
再評価手法を用いた場合の1つの動作生成にかかる計算時間を\tableref{tab:ch5_calc_time_revaluation}に示す.
また,再評価手法を用いない場合の計算時間を\tableref{tab:ch5_calc_time_140mm}に示す.
これらの表では地形ごとに,計算時間の最大値,最小値,平均,標準偏差を示しており,単位はミリ秒である.

結果より,再評価手法を用いた場合の計算時間の最大値は,用いていない場合の最大値の倍程度となっていることがわかる.
また,再評価手法を用いた場合の計算時間の平均は,用いていない場合と比べて150msec程度大きくなっていることがわかる.

\newpage

\begin{table}[htbp]
  \caption{Failure Count of Revaluation}
  \label{tab:ch5_failure_count_1}  % chktex 24
  \small
  \centering
  \begin{tabular}{|c|c|c|c|c|c|c|c|} \hline  % chktex 44
    \multirow{3}{*}{地形} & \multirow{3}{*}{グラフ探索} & \multicolumn{5}{c|}{失敗の回数} & \multirow{3}{*}{失敗率 $[\%]$} \\ \cline{3-7}  % chktex 44
     & & \multirow{2}{*}{脚接地点が} & \multicolumn{3}{c|}{脚軌道が可動範囲外を通る} & \multirow{2}{*}{総失敗} & \\ \cline{4-6}  % chktex 44
     & の回数 & 可動範囲外 & 遊脚時 & 接地時 & \begin{tabular}{c}胴体平行\\移動時\end{tabular} & 回数 & \\ \hline  % chktex 44
    平面     & 340 & 0 & 21 & 0 & 0 & 0 & 6.17 \\ \hline  % chktex 44
    上り斜面 & 700 & 0 & 35 & 0 & 0 & 0 & 5.00 \\ \hline  % chktex 44
    下り斜面 & 712 & 0 & 18 & 0 & 0 & 0 & 2.52 \\ \hline  % chktex 44
    上り段差 & 500 & 0 & 21 & 0 & 0 & 0 & 4.20 \\ \hline  % chktex 44
    下り段差 & 640 & 0 & 34 & 0 & 0 & 0 & 5.31 \\ \hline  % chktex 44
  \end{tabular}
\end{table}

\begin{table}[htbp]
  \caption{Failure Counts without Revaluation}
  \label{tab:ch5_failure_count_2}  % chktex 24
  \small
  \centering
  \begin{tabular}{|c|c|c|c|c|c|c|c|} \hline  % chktex 44
    \multirow{3}{*}{地形} & \multirow{3}{*}{グラフ探索} & \multicolumn{5}{c|}{失敗の回数} & \multirow{3}{*}{失敗率 $[\%]$} \\ \cline{3-7}  % chktex 44
     & & \multirow{2}{*}{脚接地点が} & \multicolumn{3}{c|}{脚軌道が可動範囲外を通る} & \multirow{2}{*}{総失敗} & \\ \cline{4-6}  % chktex 44
     & の回数 & 可動範囲外 & 遊脚時 & 接地時 & \begin{tabular}{c}胴体平行\\移動時\end{tabular} & 回数 & \\ \hline  % chktex 44
    平面     & 389 & 0 & 0 & 0 & 0 & 0 & 0 \\ \hline  % chktex 44
    上り斜面 & 517 & 0 & 0 & 0 & 0 & 0 & 0 \\ \hline  % chktex 44
    下り斜面 & 689 & 0 & 0 & 0 & 0 & 0 & 0 \\ \hline  % chktex 44
    上り段差 & 513 & 0 & 0 & 0 & 0 & 0 & 0 \\ \hline  % chktex 44
    下り段差 & 681 & 0 & 0 & 0 & 0 & 0 & 0 \\ \hline  % chktex 44
  \end{tabular}
\end{table}


% 地形の図
\begin{figure}[htbp]
  \begin{tabular}{cc}
    \begin{minipage}[t]{0.42\hsize}
      \begin{center}
      \includegraphics[width=1.0\linewidth,trim={50 50 50 50}, clip]{figure/chapter4/revaluation/flat.png}
      \text{(a) flat terrain}
      \end{center}
    \end{minipage} 
    &
    \begin{minipage}[t]{0.42\hsize}
      \begin{center}
      \includegraphics[width=1.0\linewidth,trim={50 50 50 50}, clip]{figure/chapter4/revaluation/130mm.png}
      \text{(b) up step terrain}
      \end{center}  
    \end{minipage}
    \\
    \begin{minipage}[t]{0.42\hsize}
      \centering
      \includegraphics[width=1.0\linewidth,trim={50 50 50 50}, clip]{figure/chapter4/revaluation/-130mm.png}
      \centering
      \text{(c) down step terrain}
    \end{minipage} 
    &
    \begin{minipage}[t]{0.42\hsize}
      \centering
      \includegraphics[width=1.0\linewidth,trim={50 50 50 50}, clip]{figure/chapter4/revaluation/15deg.png}
      \centering
      \text{(d) up slope terrain}
    \end{minipage}    
    \\
    \begin{minipage}[t]{0.42\hsize}
      \centering
      \includegraphics[width=1.0\linewidth,trim={50 50 50 50}, clip]{figure/chapter4/revaluation/-15deg.png}
      \centering
      \text{(e) down slope terrain}
      
    \end{minipage}     
    &
    \\
  \end{tabular}
  \caption{Terrain}
  \label{fig:ch5_simu_terrain} % chktex 24
\end{figure}

\newpage


\begin{table}[htbp]
  \caption{Computational Time to Generate One Operation (when Revaluating)}
  \label{tab:ch5_calc_time_revaluation}  % chktex 24
  \centering
  \begin{tabular}{|c||c|c|c|c|} \hline  % chktex 44
    \multirow{2}{*}{地形} & \multicolumn{4}{c|}{1つの動作生成にかかる計算時間} \\ \cline{2-5}  % chktex 44
     & 最大値 $[msec]$ & 最小値 $[msec]$ & 平均 $[msec]$ & 標準偏差 $[msec]$ \\ \hline \hline  % chktex 44
    平面 & 2687 & 14.75 & 655.4 & 606.7 \\ \hline  % chktex 44
    上り段差(130mm)& 2631 & 8.049 & 424.3 & 477.9 \\ \hline  % chktex 44
    下り段差(130mm)& 2761 & 10.53 & 542.7 & 545.3 \\ \hline  % chktex 44
    上り斜面(15度) & 2216 & 14.13 & 436.4 & 409.3 \\ \hline  % chktex 44
    下り斜面(15度) & 4352 & 10.39 & 435.6 & 524.3 \\ \hline  % chktex 44
  \end{tabular}
\end{table}

\begin{table}[htbp]
  \caption{Computational Time to Generate One Operation (without Revaluating)}
  \label{tab:ch5_calc_time_140mm}  % chktex 24
  \centering
  \begin{tabular}{|c||c|c|c|c|} \hline  % chktex 44
    \multirow{2}{*}{地形} & \multicolumn{4}{c|}{1つの動作生成にかかる計算時間} \\ \cline{2-5}  % chktex 44
     & 最大値 $[msec]$ & 最小値 $[msec]$ & 平均 $[msec]$ & 標準偏差 $[msec]$ \\ \hline \hline  % chktex 44
    平面 & 1366 & 16.23 & 463.4 & 351.3 \\ \hline % chktex 44
    上り段差(130mm)& 1364 & 2.975 & 375.9 & 327.2 \\ \hline % chktex 44
    下り段差(130mm)& 1344 & 10.24 & 388.6 & 347.6 \\ \hline % chktex 44
    上り斜面(15度) & 1138 & 4.629 & 271.3 & 263.4 \\ \hline % chktex 44
    下り斜面(15度) & 1286 & 2.364 & 301.9 & 267.0 \\ \hline % chktex 44
  \end{tabular}
\end{table}

\subsection{考察}
\tableref{tab:ch5_failure_count_1}より,再評価手法を用いた場合でも脚軌道生成の失敗が生じてしまうことがわかる.
しかし,脚軌道生成に失敗する場合はすべて,遊脚時に脚軌道が可動範囲外を通ることが原因である.
これは,\figref{fig:ch5_revaluation_reason}を用いて以下の通りに説明することができる.
なお,この図は\figref{fig:ch5_range_revaluation}の$100<x<200,-100<z<0$の範囲を拡大したものである.
支持脚時の脚先座標を2点,(a)(b),遊脚時の脚先座標を2点,(c)(d)としてプロットしている.

% 列挙する
\begin{enumerate}
  \item 第1関節から見た脚先位置が胴体の平行移動によって(a)から(b)へ移動する.
  \item 遊脚時に(b)を始点として,中継点(c)を通って終点(d)へ移動する際に,脚軌道生成に失敗するため再評価を行う.
  \item すでに(b)は可動範囲外にあるため,再評価をしたとしても,再び脚軌道生成に失敗する.
\end{enumerate}

これを防ぐためには,Fig.5 の\figref{fig:ch5_revaluation_reason}に脚先が移動した時点まで遡って再評価を行う必要がある.
しかし,直前の1動作のみを再評価するときと比べ,やり直す量が増えるため,計算時間がのびてしまう.
また,近似された可動範囲を狭めてしまう場合,
本来歩行可能であった地形で歩行することができなくなってしまう可能性が危惧されたが,
\tableref{tab:ch5_failure_count_2}より,最初から最小半径を$140 [mm]$とした場合に,脚軌道生成の失敗が生じていないことがわかる.
そのため,提案した再評価手法を用いずとも,最小半径を$140 [mm]$に設定することで脚軌道生成の失敗を防ぐことができると分かった.

また,計算時間について,再評価手法を用いた場合の計算時間の平均値は,用いない場合と比べて150msec程度大きくなるのみであり,
失敗時のみ再評価を行ったことで,計算時間が2倍以上に伸びることはないことがわかる.
標準偏差についても,再評価手法を用いた場合の方が大きくなっているが,大きな差はないことがわかる.
そのため,再評価手法を用いたとしても,計算時間が大幅に伸びないことが確認できた.

結論として,再評価手法によって生成された自由歩容パターンを用いても,脚軌道生成の失敗が生じてしまうことが確認された.
しかし,再評価手法を用いた場合,計算にかかる時間は大幅に伸びないことが確認された.
そのため,脚軌道生成の失敗という問題においては,再評価手法は有効でないといえる.
また,脚の近似された可動範囲における最小半径を140mmとすることで,脚軌道生成の失敗は生じないことが示された.
よって,脚軌道生成の失敗を防ぐためには,最小半径を140mmとすることが有効であるといえる.

% figure/chapter4/revaluation_reason.png
\begin{figure}[htbp]
  \centering
  \includegraphics[width=0.5\linewidth]{figure/chapter4/revaluation_reason.png}
  \caption{Reason for Failure of Revaluation}
  \label{fig:ch5_revaluation_reason} % chktex 24
\end{figure}