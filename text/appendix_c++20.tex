% C++20への移行

\chapter{C++20への移行}\label{chapter:appendix_cpp20}
\section{C++20の新機能}
C++にはコンパイラの標準規格として,C++98,C++03,C++11などが存在する.
その中でもC++11以降は約3年に一度のペースで新しい規格が策定されている.
先行研究のプログラムでは,C++17を使用していたが,本研究ではC++20\cite{Thomas_C++20}を使用するように変更を行った.
C++20では,C++17からの変更点として,以下のようなものがある.
\begin{itemize}
  \item constexpr関数の制限緩和
  \item 標準ライブラリの多くの関数がconstexpr関数に変更
  \item conceptの導入
  \item std::number,std::formatの導入
\end{itemize}
これらの機能を使用することで,プログラムの最適化を行うことができる上,プログラムの可読性を向上させることができる.
グラフ探索の計算時間の短縮は,当研究室でグラフ探索の計算のアルゴリズムの最適化の研究が数々行われていることからもわかるように,
重要な課題である.
また,先行研究のプログラムでは,C言語とC++17の機能を混在して使用していたため,
可動性が低く,新しい機能を追加することが困難であった.
よって,C++20への移行は,プログラムの可動性を向上させるためにも重要である.
以下に各機能の詳細を記述する.

\section{constexpr変数・関数}
constexpr(コンストエクスプロー・コンストイーエックスピーアール)キーワードは,C++11から導入されたキーワードであり,
変数や関数に対して使用することができる.

constexpr変数は,コンパイル時に評価される変数であり,定数として扱うことができる.
C言語における同様の機能としては,マクロがあげられるだろう.
マクロはプリコンパイル時に文字を置換するため,定数として扱うことができる.
しかし,マクロは文字を置換して置き換えるため,型を持たない.
そのため,マクロを使用することで型の不一致や意図しないキャストの発生を引き起こす可能性がある.
また,マクロはグローバルスコープで定義されるため,名前の衝突を防ぐことができない.
constexpr変数は,マクロのようにふるまうことができるが,型を持つことやスコープを限定することができる.

constexpr関数はコンパイル時に評価される関数であり,
C言語におけるマクロ関数のような処理を実現するために使用される.
たとえば,\coderef{convert_func_as_macro}のようなプログラムを考える.

\begin{lstlisting}[caption=convert func as macro,label=convert_func_as_macro]
#include <iostream>

// マクロ関数として定義する.
#define CONVERT_TO_RAD(deg) (deg * 3.1415926535f / 180.0f)  

int main()
{
  float deg = 90.0f;
  
  // CONVERT_TO_RAD(deg)はプリコンパイル時に置換される.
  // よって deg * 3.1415926535f / 180.0f に置換される.
  std::cout << CONVERT_TO_RAD(deg) << std::endl;
}
\end{lstlisting}

% 1行開ける
\vskip \baselineskip  % chktex 1

このプログラムでは,マクロ関数を使用して,度数法で表された角度をラジアンに変換している.
プログラムを記述する際にはラジアンで角度を表現すると可読性が低くなるため,
度数法で記述することによる利点は大きいが,実際の処理ではラジアンで角度を表現する必要があるので,
このようなマクロが実際に使用されることは多いだろう.

しかし,マクロにはいくつかの問題点がある.
まず1つとして,マクロは常にグローバルスコープで定義されることである.
通常C++の開発においては,クラスや名前空間を使用して,変数や関数をスコープを限定して定義することが多い.
スコープを限定することは,変数や関数の名前が衝突することを防ぐことができるため,
大規模な開発や複数のライブラリを使用する場合には必須である.
だが,マクロは名前空間内に定義したとしてもグローバルスコープに展開されるため,
名前の衝突を防ぐことができないのである.

もう1つの問題点は,マクロは型の確認を行わないことである.
C++はpythonやjava scriptなどの動的型付け言語とは異なり静的型付け言語である.
そのため,コンパイル時に型の不一致や意図しないキャストを警告として出力することができ,
ランタイムエラーを防ぐことができる.
しかし,マクロはプリコンパイル時に文字を置換するだけであるため,型の確認を行わない.
そのためたとえば上記のマクロ関数において,引数をint型やdouble型,果てはその他のクラスなどに変更しても,
float型との掛け算演算子が定義されていればコンパイルは通ってしまう.
先行研究のプログラムではマクロ関数を使用している箇所が多く存在したため,
実際に浮動小数点型のfloat型とdouble型が混在している箇所が存在した.

これをconstexpr関数を使用することで,以下のように書き換えることができる.

\begin{lstlisting}[caption=convert func as constexpr,label=convert_func_as_constexpr]
#include <iostream>

// constexpr関数として定義する.
constexpr float ConvertToRad(float deg)
{
  return deg * 3.1415926535f / 180.0f;
}

int main()
{
  // deg を constexpr変数として定義する.
  constexpr float deg = 90.0f;
  
  // ConvertToRad(deg)はコンパイル時に実行される.
  // よって 1.57079632675f に置換される.
  std::cout << ConvertToRad(deg) << std::endl;
}
\end{lstlisting}

% 1行開ける
\vskip \baselineskip  % chktex 1

このようにconstexpr関数を使用することで,マクロ関数のようにコンパイル時に評価される関数を定義することができる.
また,constexpr関数はコンパイル時に評価が行われるため,コンパイル時に型の確認を行うことができる.
加えて,constexpr関数はスコープを限定することができるため,名前の衝突を防ぐことができる.

以上のようにマクロの持つ問題点を解決することができるが,constexpr関数の本当の利点はコンパイル時に処理が実行されることである.
\coderef{convert_func_as_constexpr} の14,15行目にあるように,引数を含めてコンパイル時に評価が可能であれば,
コンパイル時に関数が実行される.そのため,展開される内容は90.0f * 3.1415926535f / 180.0fではなく,
1.57079632675fである.
このようにコンパイル時に評価が行われることで,実行時に関数は実行されないため,
実行時の計算時間を短縮することができ,マクロに比べて高速な処理が可能である.
コンパイル時に評価される影響でコンパイル時間が長くなる問題点を除けば,
constexpr関数はマクロ関数に比べて多くの利点を持っているため,
使うことができる場面では積極的に使用することが望ましい.

constexpr関数内ではその性質上,実装にいくつかの制限がある.
しかしC++20では,以下のようにconstexpr関数の制限が緩和されている.
\begin{itemize}
  \item constexpr関数内でのtry-catchブロックを許可
  \item 定数式からの仮想関数の呼び出しを許可
  \item 可変サイズをもつコンテナのconstexpr化
\end{itemize}
これによって,標準ライブラリの多くの関数やクラスのメンバ関数がconstexpr関数に変更されている.
よく使用するものでは,std::vectorのコンストラクタやstd::stringのメンバ関数などがconstexpr関数に変更されている.

\section{Concept}
C++20では,concept(コンセプト)という新しい機能が導入された.
conceptは,テンプレートの型に対する制約を定義するための機能である.
C++17以前のテンプレートでは,型に対する制約を記述するために,SFINAE(Substitution Failure Is Not An Error)を使用していた.
SFINAEで型の制約を記述する場合,プログラムの内容に対して記述が煩雑になりがちである.
また,SFINAEはコンパイルエラーが発生した場合に,エラーメッセージがわかりにくい.
conceptを使用することで,SFINAEを使用することなく,型に対する制約を記述することができる.
以下に例を示す.

\begin{lstlisting}[caption=SFINAE,label=sfinae]
#include <iostream>
#include <type_traits>

template <typename T, 
typename = typename std::enable_if<std::is_integral<T>::value>::type>
func(T t)
{
  std::cout << t << std::endl;
}

int main()
{
  func(10);
  // func(10.0);  // コンパイルエラー
}
\end{lstlisting}

\begin{lstlisting}[caption=concept,label=concept]
#include <iostream>
#include <concepts>

template <typename T>
concept Integral = std::is_integral<T>::value;

template <Integral T>
void func(T t)
{
  std::cout << t << std::endl;
}

int main()
{
  func(10);
  // func(10.0);  // コンパイルエラー
}

\end{lstlisting}

\section{std::numbers,std::format}
C++20では,std::numbers,std::formatという新しい機能が導入された.
std::numbersはいくつかの定数を提供するための機能である.
たとえば,std::numbers::piは円周率を表す定数である.
同様な機能としてC言語にはM\_PIがあるが,C言語のM\_PIはマクロであるため,型を持たない.
そのため,M\_PIを使用する際には,型の不一致や意図しないキャストの発生を引き起こす可能性がある.
std::numbersはテンプレートを使用しているため,型を持つことができる.
任意の型を持つ定数は次のように使用することができる.

\begin{lstlisting}[caption=std::numbers,label=std_number]
#include <iostream>
#include <numbers>

int main()
{
  // std::numbers::pi_vは円周率を表す定数である.  
  std::cout << std::numbers::pi_v<float> << std::endl;  // float型
  std::cout << std::numbers::pi_v<double> << std::endl;  // double型
}

\end{lstlisting}

% 1行開ける
\vskip \baselineskip  % chktex 1

std::formatは,文字列をフォーマットするための機能である.
C言語では,printf関数を使用して文字列をフォーマットすることができるが,
printf関数はパラメーターの受け渡しをスタックで行う為,
printf内の置換指定子と不整合のあるパラメーターを指定するとプログラムの実行時にエラーが発生する.
std::formatはprintfとほぼ同じような機能を提供するが,パラメータが合わない場合は例外を投げる.
以下に例を示す.

\begin{lstlisting}[caption=std::format,label=std_format]
#include <iostream>
#include <format>

int main()
{
  // std::formatは文字列をフォーマットするための機能である.
  std::string str = std::format("Hello, {}!", "world");
  std::cout << str << std::endl;
}

\end{lstlisting}
