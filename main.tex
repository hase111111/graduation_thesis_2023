%%%%%%%%%%%%%%%%%%%%%%%%%%%%%%%%%%%%%%%%%%%%%%%%%%%%%%%%%%%%%%%%%%%%%%%%
%%
%% main.tex
%% LaTeX-2e 専用
%% 
%% 
%%        設計工学研究室 学位論文テンプレート
%%
%%                      作成日時    2018年 10月 1日
%%
%%%%%%%%%%%%%%%%%%%%%%%%%%%%%%%%%%%%%%%%%%%%%%%%%%%%%%%%%%%%%%%%%%%%%%%%

\documentclass[12pt, a4paper, titlepage]{jsbook}      %ドキュメントクラス

%パッケージ
\usepackage[dvipdfmx]{graphicx, color}
\usepackage{amsmath}
\usepackage{amssymb}
\usepackage{array}
\usepackage{hhline}
\usepackage{afterpage}
\usepackage{enumerate}
\usepackage{multicol}
\usepackage{fancyhdr}
\usepackage{subfigure}
\usepackage{bm} %ベクトル表示
\usepackage{slashbox}

%新定義コマンド
\newcommand{\sankou}[1]{$ ^{\text{[#1]}} $}
\newcommand{\chref}[1]{第~\ref{#1}~章}
\newcommand{\secref}[1]{\ref{#1} (p.\pageref{#1})}
\newcommand{\tableref}[1]{(Table.\ref{#1})}
\newcommand{\figref}[1]{(Fig.\ref{#1})}
\newcommand{\figpref}[1]{(Fig.\ref{#1} p.\pageref{#1})}
\newcommand{\degree}{\char'27\kern-.3em\hbox{C}} 

%再定義コマンド
\renewcommand{\figurename}{Fig.~}
\renewcommand{\tablename}{Table.~}
\renewcommand{\labelitemi}{・}
\renewcommand{\labelenumi}{(\theenumi)}

\renewcommand{\eqref}[1]{式 (\ref{#1})}

\renewcommand{\subfigtopskip}{0mm}%図と図の間
\renewcommand{\subfigcapskip}{0mm}%図と副題の間

%余白変更
\setlength{\textwidth}{\fullwidth}
\setlength{\evensidemargin}{\oddsidemargin}

\title{学位論文のテンプレート}
\author{埼玉大学 工学部 機械工学科
        \\
        埼玉 太郎
        }

\begin{document}

    %タイトル
    %%%%%%%%%%%%%%%%%%%%%%%%%%%%%%%%%%%%%%%%%%%%%%%%%%%%%%%%%%%%%%%%%%%%%%%%
%%
%% 題目.tex
%% LaTeX-2e 専用
%% 
%% 
%%        設計工学研究室 学位論文テンプレート
%%
%%                      作成日時    2020年 2月 11日
%%
%%%%%%%%%%%%%%%%%%%%%%%%%%%%%%%%%%%%%%%%%%%%%%%%%%%%%%%%%%%%%%%%%%%%%%%%
\thispagestyle{empty}
\newcommand{\ctext}[1]{\textcolor[rgb]{0.65, 0.65, 0.65}{\raise0.2ex\hbox{\textcircled{\scriptsize{#1}}}}}
%% 卒業論文題目     
\begin{center}
        {\huge 埼玉大学 工学部} \\
        {\huge 機械工学科} \\        
        \vspace{10mm}
        {\Huge 令和5年度 \quad 卒業論文}\\
        \vspace{10mm} 
        
        {\Huge グラフ探索を用いた多脚ロボットの歩容パターン生成における脚軌道生成失敗時の歩容パターンの再評価手法} \\
        \vspace{10mm}
        {\LARGE A Reevaluation Method of Gait Pattern Generation for Multilegged Robots Using Graph Search when Leg Trajectory Generation Fails} \\        
        \vspace{10mm} %題目の長さに応じて適宜修正すること
 \end{center}
        
 \begin{table}[!h]
        \begin{flushright}        
        \renewcommand{\arraystretch}{1.5}
        \begin{tabular}{|c|cc|}
            \hline
            {\Large 学科長}    & {\Large  荒居善雄 教授 \quad} & {\Large\ctext{印}}\\        
            \hline
            {\Large 主指導教員} & {\Large  琴坂信哉 准教授 } &{\Large\ctext{印}}\\
            \hline
            {\Large 副指導教員} & {\Large  程島竜一 准教授 }& \\
            \hline
        \end{tabular}
        \renewcommand{\arraystretch}{1.0}        
    \end{flushright}        
 \end{table}
\begin{table}[!b]
        \begin{flushright}
        \renewcommand{\arraystretch}{1.2}
        \begin{tabular}{|c|c|}
            \hline
             {\Large 提出日} & {\Large 2023年2月XX日}\\ %年は西暦で記載
             \hline
             {\Large 研究室} & {\Large 設計工学 }\\
             \hline
             {\Large 学籍番号} & {\Large 20TM028}\\
             \hline
             {\Large 氏  名} & {\Large 長谷川 大晴}\\
            \hline
        \end{tabular}
        \renewcommand{\arraystretch}{1.0}
        \end{flushright}
\end{table}
                                %タイトル ※修士は題目_修士、学士は題目_学士に変更すること。編集するファイルもそれぞれ対応するファイルを編集する。

    \frontmatter
    \tableofcontents
    \listoffigures
    \listoftables
    \mainmatter{} 
    %!!重要!!
    %各章は奇数ページから開始するように調整すること
    %つまり各章は偶数ページで終わるように、改ページ等を使用して調整する。
    
\chapter{序論}\label{chapter:序論}
第\ref{chapter:序論}章では,本研究の背景と先行研究,そして研究の目的を述べる.


\section{背景}

% 流れ
% 多脚ロボットの特徴と,不整地の移動に適していることを述べる
% 不整地の移動において,6脚であることによる利点を述べる
% 適応歩容の必要性を述べる
% 適応歩容の実現方法を述べる,そして,グラフ探索による手法を述べる

% 1.1.1 章
\subsection{不整地における多脚ロボットの活用}
近年,人間に代わって作業を行う移動ロボットの導入が進められている\cite{Sotnik_Prospects_for_Introduction}.
Pudu社が開発したロボットのBellaBot\cite{Pudu_BellaBot}が,
レストランで配膳の作業を行う姿は一般に見ることができるようになった.
これらのロボットは人間が移動を行う空間での使用を前提としており,多くはタイヤやクローラを用いて移動を行うが,
その他の移動様式として,脚を使用して移動を行うロボット(以下脚ロボット)が存在する.
脚ロボットは他の移動様式を用いて移動するロボットに比べて,
以下に示すような利点がある\cite{Locomotion_for_difficult_terrain}.

\begin{itemize}
  \item 障害物をまたいで移動できるため,対地適応性が高い
  \item 脚接地点を離散的に選択できるため,環境に与える影響が小さい
  \item スリップすることなく全方向に移動できる
\end{itemize}

障害物をまたいで移動できることにより,脚ロボットはタイヤでは移動できないような凹凸が激しい地形や,不連続な地形においても移動することが可能である.
また,砂利で舗装された道のような,クローラではスリップしてしまうような環境においても移動することが可能である.
この特徴を生かして,実際に図\ref{fig:nedo_spot}のようにBoston Dynamics Inc. によって開発された4足歩行ロボットのSpot\cite{Boston_Dynamics_Spot}を用いて,
林業を行う山間地で作業を行う実証実験が行われている\cite{NEDO}.
山間地は斜面である上に,伐根作業によって木の根が残っているため凹凸が激しく,タイヤやクローラでは移動が困難である.
しかし,脚ロボットであるSpotは障害物をまたいで移動することができるため,このような環境での作業に適しており,
導入による人手不足の解消や,作業効率の向上が期待されている.

また,離散的に脚接地地点を選択できるため,タイヤやクローラによる移動と比較して環境に与える影響が小さい.
この特徴を生かしたロボットの実例として,韓国海洋科学技術院によって開発された6脚ロボットのCrabster\cite{J_Kim_Dexterous_Crabster}があげられる.
図\ref{fig:crabster}に示したCrabsterは,流れの早い海底での作業を想定して開発されており,
6本の脚による移動を行うことや,4本の脚を海底に接地させ,残りの2本の脚を用いて作業を行うことが可能である.
脚を用いた移動では海底の砂を大量に巻き上げることがないため\cite{J_Kim_Little_Crabster},カメラを用いた観察や,センサを用いた地形の計測に優れている.
以上より,脚ロボットは不整地での使用に適していると言える.

\begin{figure}[htbp]
  \begin{center}
    \includegraphics[width=100mm, clip]{figure/chapter1/NEDO.png}
   \caption{Demonstration Experiment with Spot\cite{NEDO}}
   \label{fig:nedo_spot}
  \end{center}
\end{figure}

\begin{figure}[htbp]
  \begin{center}
    \includegraphics[width=100mm, clip]{figure/chapter1/crabster.png}
   \caption{Crabster\cite{J_Kim_Dexterous_Crabster}}
   \label{fig:crabster}
  \end{center}
\end{figure}

脚ロボットの歩行形態は大別して,動歩行と静歩行に分けられる.
静歩行とは重心の位置が常に接地した脚を頂点とする多角形内に存在するように歩行を行うことであり,

不整地を歩行する際は,ロボットが転倒することのないように常に静歩行を行うことが求められる.

2脚ロボットの場合,歩行の際には常に1脚が支持脚となり,もう1脚が遊脚となることから,
静歩行を行うためには重心の位置を脚先の足裏面積の内側に入るように制御する必要がある.
そのため,障害物をまたいで移動すること

ロボットの性能の指標として,歩行速度や消費エネルギーなどがあるが,不整地において用いることを考え,
静的安定余裕\cite{Hirose_Static_stability_criterion}(Stability Margin)を指標として分別することとする.
静的安定余裕とは,ロボットが静的に安定するために必要な脚位置と重心位置の関係を表す指標である.
支持脚を結んでできる多角形の重心が,支持脚の内部にある場合,ロボットは静的に安定する.
そのため,静的安定余裕では多角形の辺から重心までの距離を評価に用いる.
以上より,6

しかし,実際に不整地において多脚ロボットが活用された事例は少ない.
これは,多脚ロボットが以下に示すような問題を抱えているためである\cite{Locomotion_for_difficult_terrain}.

\begin{itemize}
  \item 脚の機構はアクチュエータの数が多いため,重量が大きくなってしまう
  \item 移動には歩行のための制御が必要であり,タイヤやクローラに比べて制御が複雑になる
\end{itemize}

多脚ロボットの汎用的な活用を考えると,脚の機構の軽量化と,歩行のための制御の手法が必要である.
そこで本研究では多脚ロボットの制御に着目し,不整地における多脚ロボットの活用のための制御手法を論じる.

% 1.1.2 章
\subsection{固定歩容と自由歩容}
多脚ロボットが歩行を行う際には,脚を適切な順番で動かす必要がある.

歩容にはさまざまな種類があるが,大きく分別するとカムやリンクを用いて,
周期的に脚を動かす固定歩容と,非周期的に脚を動かす自由歩容がある.

\subsection{グラフ探索による自由歩容パターン生成手法}
自由歩容パターンを生成する手法として,グラフ探索による自由歩容パターン生成手法がある.
これは,脚位置や動作を離散化することでロボットの歩容をグラフに落としこみ,
そのグラフを探索することによって数動作先までの歩容パターンの組み合わせを網羅的に調べ,最適な歩容パターンを選択する手法である.
この手法の特徴として,数動作先までを考慮して歩容パターンを生成するため,デッドロックに陥りにくいという点や,
効率的な歩容パターンを生成することができるという点が挙げられる.

しかし、グラフ探索による自由歩容パターン生成手法は,脚の本数が増えることで脚の動かし方の組み合わせが増えるため,
グラフの規模が大きくなり,実時間内の計算が困難になるという問題がある.
4脚ロボットにおいて,静的安定性を保ちながら歩行する場合,1度に遊脚することができる脚は1本であるため,
実時間内の計算は容易である\cite{Prabir_Graph_search}.
だが,6脚ロボットは静的安定性を保ちながら最大3本の脚を遊脚することができるため,
グラフの規模が大きくなり,実時間内の計算が困難になる.
この問題を解決するためにPrabirらは歩容パターンをウェーブ歩容に限定することで,
実時間内の計算を可能にした\cite{Prabir_Graph_search_Six}.

そこで本研究室ではロボットの動作を制限することなく,グラフ探索による自由歩容パターン生成手法を6脚ロボットに適用するため,
離散化された脚位置の組み合わせを利用してグラフの階層構造化を行った.
また,自由歩容パターン生成による接地地点の計算と脚軌道の生成を分離した.
これらによって,グラフ探索による自由歩容パターン生成手法を6脚ロボットに適用することが可能になった.

本研究室では,ロボットを動作させる地形やロボットの動作によって段階的に開発を行っており,
これまでに2次元空間において,直進動作\cite{Oki_Graph_search},目的姿勢での停止\cite{Nakaoka_Graph_search},
旋回動作\cite{Shina_Graph_search}を行うための歩容パターン生成手法の実装に成功した.
また3次元空間においても,脚位置の離散化方法を3次元に適応させたことで\cite{Miura_Graph_search},
直進動作\cite{Hato_Graph_search}を行うための歩容パターン生成手法の実装に成功した.

\section{本研究の目的}
これまでの研究によって,3次元の不整地において,重心高さを変更しつつ,
自由歩容パターン生成を行うことが可能となった.
しかし低頻度ではあるが,グラフ探索に成功したとしても脚軌道が生成できず,その歩容パターン通りに歩行することできなくなり,
動作を停止してしまう問題が生じてしまった.

これは,グラフ探索と脚軌道生成を分離したことによって生じた問題であり,
先行研究では継続的にこの問題について言及されていたが,
グラフの規模を大幅に増大させることなく,また歩容生成の成功率を低下させることなく,
この問題を解決することはできなかった.

そこで本論文では,常に脚軌道生成に成功するような歩容パターン生成手法を提案し,
脚軌道生成の失敗による動作停止を防ぐことを目的とする.

\section{本論文の構成}
本論文は,全6章から構成される.

第2章「歩容パターンの再評価手法の提案」では,常に脚軌道生成が可能になる手法として,
歩容パターンの再評価手法を提案し,その機能を述べる.

第3章「歩容パターンの再評価手法の実装」では,提案したプログラムの実装方法を述べる.

第4章「再評価手法の有効性の確認のための歩行シミュレーション」では,
提案手法を用いたシミュレーション実験の結果を述べる.

第5章「常に脚軌道生成が可能な自由歩容パターン生成手法を用いた実機による歩行実験」
では,提案手法を用いた実機試験の結果を述べる.

第6章「結論」では本論文の結論と今後の課題を述べる.

    
%% 歩容パターンの再評価手法の提案.tex
%% LaTeX-2e 専用

%% 全体の流れとしては,まず,先行研究の問題点を指摘し,次に,歩容パターンの再評価手法を提案する.

\chapter{歩容パターンの再評価手法の提案}\label{chapter:歩容パターンの再評価手法の提案}
第\ref{chapter:歩容パターンの再評価手法の提案}章では,先行研究の手法およびその問題点を指摘し,
常に脚軌道生成が可能な自由歩容パターン生成手法として,歩容パターンの再評価手法を述べる.

% 先行研究の章
\section{本研究室における自由歩容パターン生成の先行研究}
最初にグラフ探索よる自由歩容パターン生成手法において用いる用語を定義し,
先行研究で行われてきた自由歩容パターン生成手法について述べる.
また,先行研究で用いられてきた自由歩容パターン生成手法の問題点について述べる.

\subsection{グラフ理論について}
本論文ではグラフ理論を用いた自由歩容パターン生成手法を論ずるため,まずはグラフ理論について説明をする.
グラフとは,頂点(ノード)とそれらを結ぶ辺(エッジ)からなる図形である.
このグラフを用いて,さまざまな問題を取り扱う学問をグラフ理論という.

以降の説明を簡単にするため,この論文で用いるグラフ理論の用語について簡潔に述べる.
グラフ上のあるノードから別のノードにエッジを用いて移動することを遷移と呼ぶ.
遷移の際,移動前のノードを始点,移動後のノードを終点と呼ぶ.
グラフの種類は大別して有向グラフと無向グラフに分けられ,
エッジに向きがあるものを有向グラフ(\figref{fig:directed_graph}),
逆に向きを持たないものを無向グラフ(\figref{fig:undirected_graph})という.
また,閉路を持たず,かつ,すべてのノード間にエッジが存在するグラフを木という.
このような木構造をもつグラフのうち,\figref{fig:tree_graph}のように,
根となるノードを持ち,そのノードからすべてのノードに到達可能なものを根付き木という.
根付き木を図形として表現する場合は簡単のため,
\figref{fig:tree_graph}のように根を上部に配置することが多い.

根付き木には無向グラフと有向グラフの2種類が存在するが,
後述する歩行パターングラフは有効グラフであるため,
有向の根付き木について説明を行う.
根付き木のエッジが,根が始点となるように伸びている場合,
あるノードから遷移可能なノードをそのノードの子ノードと呼ぶ.
逆に,あるノードに遷移可能なノードをそのノードの親ノードと呼ぶ.
親ノードを持たないノードを根ノードと呼び,子ノードを持たないノードを葉ノードと呼ぶ.
また,あるノードから根ノードまでのエッジの数をそのノードの深さと呼び,
根ノード自身の深さは0となる.

たとえば\figref{fig:tree_graph}において,ノードAが根ノードであり,ノードB,Cがその子ノードである.
また,ノードB,ノードD,E,ノードCはノードFを子ノードとして持ち,ノードD,E,Fは葉ノードである.
ノードAの深さは0であり,ノードB,Cの深さは1,ノードD,E,Fの深さは2となる.

\begin{figure}[h]
  \subfigure[Undirected Graph]{%
  \label{fig:undirected_graph}
  \includegraphics[width=0.48\columnwidth]{figure/chapter2/undirected_graph.png}}
  \hspace{0.02\columnwidth}
  \subfigure[Directed Graph]{%
  \label{fig:directed_graph}
  \includegraphics[width=0.48\columnwidth]{figure/chapter2/directed_graph.png}}
  \caption{Examples of simple graphs}
  \label{fig:example_simple_graphs}  % chktex 24
\end{figure}

\begin{figure}[h]
  \begin{center}
    \includegraphics[width=50mm, clip]{figure/chapter2/tree_graph.png}
    \caption{Tree Graph}
    \label{fig:tree_graph} % chktex 24
  \end{center}
\end{figure}

グラフのあるノードから別のノードに到達するための経路をパスと呼び,
これを求めることをグラフ探索と呼ぶ.
グラフ探索には,深さ優先探索,幅優先探索などのさまざまなアルゴリズムが存在する.
深さ優先探索では,始点となるノードから,深さが深くなる方向を優先して探索を行う.
これに対して,幅優先探索では,始点となるノードから,深さが浅いノードを優先して探索を行う手法である.

\subsection{歩容パターングラフの定義}
本研究においては,6脚ロボットの歩容パターンをグラフを用いて表現する.
グラフはロボットの状態をノードとし,ロボットの状態間の遷移,つまりロボットの動作をエッジとして作成する.
グラフは有向の根付き木とし,現在のロボットの状態を根ノード,
その姿勢から1動作で到達できる姿勢を子ノードとして根ノードに接続する.
また,このようにして作られたグラフを歩容パターングラフと定義する.
\figref{fig:image_of_gait_pattern_graph}に歩容パターングラフのイメージを示した.

本手法では,まず歩容パターングラフを作成する.
そして,根ノードから最適な動作を行う葉ノードまでのパスをグラフ探索によって求め,
そのパスに含まれる深さ1のノードを次の動作としてロボットに実行させる.
これを繰り返すことで,ロボットの歩容パターンを生成しているのである.

グラフ探索による歩容パターン生成においては,網羅的にロボットの状態を調べ上げるため,
実時間内の計算を行うにはグラフの規模を小さくすることが求められる.
しかし,歩容パターングラフはロボットの状態や動作を対象とするため無数の組み合わせが存在し,
そのすべてを網羅的に調べ上げることは困難である.
そのため,状態や動作を離散化することで歩容パターン生成をグラフへ落とし込む必要がある.
以下に各要素の離散化方法について述べる.
\\  % 1行開ける

\begin{figure}[h]
  \begin{center}
    \includegraphics[width=100mm, clip]{figure/chapter2/gait_pattern_graph.png}
    \caption{Image of Gait Pattern Graph}
    \label{fig:image_of_gait_pattern_graph} % chktex 24
  \end{center}
\end{figure}

\subsubsection{グラフの階層構造}
前述のとおり,ロボットの脚位置は脚の可動範囲内であれば,無数の位置を取ることができる.
そのため本手法では,基準となる脚位置を決め,その基準からの相対位置を用いて脚位置を離散化している.
Prabirらが提案した手法では2次元平面での移動を前提としていたが\cite{Prabir_Graph_search},
これを三浦が3次元空間へ拡張した\cite{Miura_Graph_search}.

\figref{fig:discretization}に支持脚の脚位置の離散化の様子を示した.
\figref{fig:discretization}のように脚位置の基準座標を4とし,
脚位置4と同じ高さにあるかつ,進行方向に対して基準位置よりも前方にある脚位置を6,後方にある脚位置を2とする.
また,脚位置6よりも高い位置にある脚位置を7,低い位置にある脚位置を5とし,
脚位置2よりも高い位置にある脚位置を3,低い位置にある脚位置を1とする.
このようにして,脚位置を7つに離散化している.
遊脚している脚の脚位置は,支持脚の脚位置1$\sim$7に対応させ,脚位置1'$\sim$7'とする.% $\sim$ で波線を引く

これにより,脚位置1$\sim$7から脚位置1'$\sim$7'への遷移によって,脚の上下運動を表現することができる.
また,脚位置1'$\sim$7'内での遷移によって,遊脚の水平方向の移動を表現することができる.
以上より,脚の上下運動と脚の水平方向の移動をグラフで表現することができることを示した.

\begin{figure}[h]
  \begin{center}
    \includegraphics[width=75mm, clip]{figure/chapter2/discretization_of_leg_pos.png}
    \caption{Discretization of Leg Posistion}
    \label{fig:discretization} % chktex 24
  \end{center}
\end{figure}

このような脚位置の離散化を行うことで,脚位置の組み合わせを有限個に抑えることができるが,
脚位置の組み合わせは未だ,$7^6 = 117649 \approx 10^5$通り存在し,
遊脚である時を含めるとさらに組み合わせが増えてしまうことがわかる.
プログラムの実行環境によって左右されるが,
プログラミング言語のC++で作成した処理では約1秒間に$10^8$回程度の計算が可能であるとされており\cite{Program_Challenge_Book},
この組み合わせをすべて歩容パターングラフに追加した場合,実時間内の処理を行うにはグラフの規模が大きくなりすぎる.
そこで,大木らは脚位置の組み合わせを階層構造化することで,
遊脚時の脚位置1'$\sim$7'を省略し,探索するノード数を減らすことに成功した\cite{Oki_Graph_search}.

階層構造とこれを利用した探索の方法を説明するために,遊脚の組み合わせと脚位置の組み合わせについて定義を行う.
遊脚の組み合わせは,ロボットの各脚について,その脚が遊脚であるか支持脚であるかを表すノードの要素である.
6脚ロボットの右前脚を1番目の脚として,時計回りに2番目の脚から6番目の脚とする.
この時,i番目の脚が支持脚であることを$v_i = 1$,遊脚である時を$v_i = 0$とすると,
遊脚の組み合わせ$V$は\eqref{eq:leg_com}のように表すことができる.

\begin{equation}\label{eq:leg_com}
  V = \{v_1, v_2, v_3, v_4, v_5, v_6\}
\end{equation}

\noindent 遊脚の組み合わせは$2^6 = 64$通り存在するが,
6脚,5脚,4脚が遊脚である組み合わせや,
隣り合う3脚が遊脚である組み合わせは
実際には取りえない組み合わせであるため,
探索するべき組み合わせは$2^6 - {}_6 \mathrm{C}_6 - {}_6 \mathrm{C}_5 - {}_6 \mathrm{C}_4 - 6 = 36$通りとなる.

また,脚位置の組み合わせとは離散化した脚位置において,各脚がどの位置にあるかを表すノードの要素である.
i番目の脚が脚位置jにあることを$k_i = j$とすると,
脚位置の組み合わせ$K$は\eqref{eq:leg_pos}のように表すことができる.

\begin{equation}\label{eq:leg_pos}
  K = \{k_1, k_2, k_3, k_4, k_5, k_6\}
\end{equation}

\noindent\eqref{eq:leg_com},\eqref{eq:leg_pos}より
ロボットの脚の状態は遊脚の組み合わせ$V$と脚位置の組み合わせ$K$を用いて表すことができるようになった.
i番目の脚の状態を$l_i = v_i \cdot k_i$とすると,$L$は\eqref{eq:leg_state}のように表すことができる.

\begin{equation}\label{eq:leg_state}
  L_{ij} = \{v_1 \cdot k_1, v_2 \cdot k_2, v_3 \cdot k_3, v_4 \cdot k_4, v_5 \cdot k_5, v_6 \cdot k_6\}
\end{equation}

\noindent\eqref{eq:leg_state}より,脚位置の組み合わせが$K = \{1,1,1,1,1,1\}$で
遊脚の組み合わせが$V = \{0,1,0,1,0,1\}$である時の脚の状態を$L_{ij} = \{0,1,0,1,0,1\}$と表すことができる.
同様に,脚位置の組み合わせが$K = \{3,1,3,1,3,1\}$で遊脚の組み合わせが$V = \{0,1,0,1,0,1\}$である時の脚状態は
$L_{ij} = \{0,1,0,1,0,1\}$と表すことができる.
このことから,ある脚位置の組み合わせから,別の脚位置の組み合わせへの遷移は,
脚の状態$L$が等しいときのみ可能であることがわかる.

以上の定義より,階層構造と階層を用いたグラフの探索方法について説明することができる.
階層とは脚位置の組み合わせ$K$が等しく,かつ,遊脚の組み合わせ$V$が異なるノードの集合と定義され,
遊脚の組み合わせが36通り存在することから,同じ階層内のノードは36個存在する.
脚の上下運動を実現したい場合は,遊脚の組み合わせ$V$が異なるノードを探索する必要があるため,
図\ref{fig:hierarchy2}のように階層内のノード36個のみを探索すればよい.

また,脚の水平運動を実現したい場合は,脚位置の組み合わせ$K$が異なるノードを探索する必要があるため,
図\ref{fig:hierarchy}のように脚の状態$L$が等しくなるノードのみを探索すればよい.
脚の状態$L$が等しくなるノードは,最大で3脚が遊脚しているときの$7^3 = 343$個であるため,
十分に実時間内の計算が可能である.

\begin{figure}[htbp]
  \begin{center}
    \includegraphics[width=75mm, clip]{figure/chapter2/hierarchy2.png}
    \caption{Search in the Same Hierarchy}
    \label{fig:hierarchy2} % chktex 24
  \end{center}
\end{figure}

\begin{figure}[htbp]
  \begin{center}
    \includegraphics[width=75mm, clip]{figure/chapter2/hierarchy.png}
    \caption{Search in the Difficult Hierarchy}
    \label{fig:hierarchy} % chktex 24
  \end{center}
\end{figure}

\subsubsection{脚軌道生成の分離}
次に歩容パターングラフのエッジについて述べる.
歩容パターングラフにおいて,エッジはロボットの動作を表す.
ロボットの動作は脚の接地・遊脚運動と,重心の移動からなるため,これらの動作に対応するエッジを作成する.
具体的には,脚の上下移動のエッジ,脚の水平移動のエッジ,重心の上下移動のエッジ,
重心の水平移動のエッジ,そして,重心の回転のエッジを用いてロボットの動作を表現する.

これらのエッジには,移動前と移動後のノードを補完するための状態を持っておらず,
単純に移動前と移動後のノードを結ぶのみである.
これはつまり,歩容パターングラフを生成するプログラムと,脚軌道を生成するプログラムが分離されていることを意味する.
脚軌道を考える場合,ロボットの関節の可動範囲を考慮する必要があり,逆運動学解を用いる脚の関節角度の計算が求められる.
しかし,逆運動学の計算には計算負荷の大きい逆三角関数の計算が必要となり,各エッジについて網羅的に計算を行うことを考えると,
実時間内の計算が困難になってしまう.
そのため本手法では,歩容パターングラフを生成するプログラムを分離し,
歩容パターングラフの生成時には脚の可動範囲は近似的な値を用いて計算することで,
実時間内の計算を可能にしている.

近似された脚の可動範囲の求め方について述べる.
近似された脚の可動範囲は脚の付け根を中心とする環状の扇形として表現する.
簡単のため,以降は扇形の外径を最大半径,内径を最小半径と呼ぶことにする.
まず,脚先を届くことができる範囲を求めるために,
最大半径を以下の手順で求める.
\begin{enumerate}
  \item 脚の付け根と脚先が同じ高さになるように脚先を上げる.
  \item 図\ref{fig:leg_range_a}のように水平方向に脚先を$1 [mm]$ずつ伸ばす.
  \item 脚先を伸ばすことができなくなった場合,\\
        図\ref{fig:leg_range_b}のように脚先と脚の付け根の高さ方向の距離の差をインデックスとして,\\
        脚先と第1関節の水平方向の距離を最大半径として記録する.
  \item 脚先を$1 [mm]$下げて(2)(3)の処理を繰り返す.脚先を下げることができなくなった場合,処理を終了する.
\end{enumerate}
次に最小半径をロボットの脚長などのパラメータから決定する.
先行研究では実験で用いたロボットにあわせ,$120 [mm]$とした.
そして,扇形の中心角を隣の脚と干渉しないように決定する.
こうして求められる近似された脚の可動範囲のイメージを\figref{fig:approximated_range_of_motion}に示す.
\figref{fig:approximated_range_of_motion}では右中央の脚の近似された脚の可動範囲のみを示しており,
赤い領域で示される図形が近似された脚の可動範囲である.

このように近似を行ったことで,以下に示す手順で脚先が脚の可動範囲内にあるか判定することができる.
\begin{enumerate}
  \item 脚の付け根から脚先までの水平方向のベクトルを求め,ベクトルの傾きが扇形の中心角の範囲内にあるか判定する.
  \item 脚の付け根から脚先までの水平方向の距離を求める.
  \item 最小半径よりも大きく,最大半径よりも小さいか判定する.
  \item 以上の条件を満たす場合,脚先は脚の可動範囲内にあると判定する.
\end{enumerate}
(2)(3)の手順は四則演算のみで計算が可能であり,
(1)の手順もベクトルの内積を用いて計算が可能であるため同様に四則演算のみで計算が可能である.
そのため,脚先が脚の可動範囲内にあるかの判定を高速に行うことができる.
本手法では脚軌道生成だけでなく,脚の接地判定にもこの近似的な脚の可動範囲を用いている.


\begin{figure}[htbp]
  \subfigure[Horizontal Movement of the Leg Tips]{%
  \label{fig:leg_range_a}
  \includegraphics[width=0.48\columnwidth]{figure/chapter2/leg_range.png}}
  \hspace{0.04\columnwidth}
  \subfigure[Determination of the Maximum Radius]{%
  \label{fig:leg_range_b}
  \includegraphics[width=0.48\columnwidth]{figure/chapter2/max_leg_range.png}}
  \caption{Caluculation of the Maximum Radius}
  \label{fig:leg_range}  % chktex 24
\end{figure}

\begin{figure}[htbp]
  \subfigure[Side View]{%
  \label{fig:side_view}
  \includegraphics[width=0.48\columnwidth]{figure/chapter2/approximated_range_motion_top.png}}
  \hspace{0.04\columnwidth}
  \subfigure[Top View]{%
  \label{fig:top_view}
  \includegraphics[width=0.48\columnwidth]{figure/chapter2/approximated_range_motion_back.png}}
  \caption{Approximated Range of Motion}
  \label{fig:approximated_range_of_motion}  % chktex 24
\end{figure}

\subsection{脚軌道生成の失敗}
先行研究ではグラフの階層構造化および脚軌道生成の分離によって,実時間内の計算が可能になった.
しかし,実際に実機を用いて歩行実験を行ったところ,低頻度ではあるが脚軌道生成に失敗することがあった.
失敗の原因として,脚の可動範囲に近似的な値を用いていることが予想される.
近似的な脚の可動範囲の境界近くは,実際には脚の可動範囲外であるが,
近似的な値を用いているために脚の可動範囲内と判定されてしまっている領域があると考えられる.
そのため,脚軌道や脚の接地点が脚の可動範囲外になってしまい,脚軌道生成に失敗することがあると考えられる.

失敗の原因についてより具体的に議論するために,脚の可動範囲の解析を行ったため,
その方法と結果を示す.
解析の対象とするロボットは,Trossen Robotics社のPhantomX Mark I\hspace{-1.2pt}I \cite{cita:phantom_x_mark_2}  % chktex 2
(以下PhantomX)とする.
PhantomX(\figref{fig:phantomx_mk2})は6脚ロボットであり,各脚に3つのアクチュエータを持つ.
また,関節配置は脚の付け根からヨー・ピッチ・ピッチの順である.

PhantomXの脚の可動範囲を求めるために,間接角度の逆運動解を求める式を導く.
まず,\figref{fig:leg_coordinate_axis}にPhantomXの脚の座標系を示す.
第1関節を原点とし,
x軸をロボットの前方方向にとり,z軸をロボットに垂直で上向きにとる.
y軸は右手座標系となるように設定する.
また簡単のため,解析には\figref{fig:joint_and_link}のようにPhantomXのアクチュエータの回転軸を結んだ仮想的なリンクを用いる.
リンク名は第1関節からCoxa Link,Femur Link,Tibia Linkであり,関節名はCoxa Joint,Femur Joint,Tibia Jointである.
リンク長は第1関節から第3関節までそれぞれ$L_c$,$L_t$,$L_f$とし,
間接角度をそれぞれ$\theta_c$,$\theta_t$,$\theta_f$とする.
間接の座標は$P_c$,$P_t$,$P_f$とし,脚先の座標を$P_{end}$とする.
$P_{end} = \{x_{end},y_{end},z_{end}\}$の時の
このとき$P_c$,$P_f$,$P_t$,$\theta_c$,$\theta_f$,$\theta_t$を
求める式は\eqref{eq:theta_c}から\eqref{eq:theta_t}である.

\begin{align}
  P_c &= \{0,0,0\} \label{eq:theta_c} \\
  \theta_c &= \arctan \frac{y_{end}}{x_{end}}  \\
  P_f &= \{L_c \cos \theta_c, L_c \sin \theta_c, 0\}  \\
  \theta_f &= \arctan \frac{z_{end}}{\sqrt{x_{end}^2 + y_{end}^2} - L_c} + 
  \arccos \frac{L_t^2 + L_f^2 - x_{end}^2 - y_{end}^2 - z_{end}^2}{2 \cdot L_t \cdot L_f} \\
  P_t &= \{(L_c + L_t \cos \theta_f)\cos \theta_c ,
                  L_t \cos \theta_f \sin \theta_c , 
                  L_t \sin \theta_f\}  \\
  \theta_t &= \arctan \frac{z_{end} - z_t}{\sqrt{x_{end}^2 + y_{end}^2} - 
              \sqrt{x_t^2 + y_t^2}} - \theta_f \label{eq:theta_t} 
\end{align}

\newpage

% phantomx mk - 2 の図
\begin{figure}[h]
  \begin{center}
    \includegraphics[width=60mm, clip]{figure/chapter2/phantomx_mk2.jpg}
    \caption{PhantomX Mark I\hspace{-1.2pt}I}
    \label{fig:phantomx_mk2} % chktex 24
  \end{center}
\end{figure}

% 脚の座標系
\begin{figure}[h]
  \begin{tabular}{cc}
    \begin{minipage}{0.5\textwidth}
      \centering
      \includegraphics[width=1.0\linewidth]{figure/chapter2/coordinate_axis.png}
      \caption{Leg Coordinate Axis}
      \label{fig:leg_coordinate_axis} % chktex 24
    \end{minipage}
    \begin{minipage}{0.5\textwidth}
      \centering
      \includegraphics[width=1.0\linewidth]{figure/chapter2/link_joint.png}
      \caption{Joint and Link Name} 
      \label{fig:joint_and_link} % chktex 24
    \end{minipage}
  \end{tabular}
\end{figure}

% 脚長の表
\begin{table}[h]
	\caption{Link Length of PhantomX}
	\label{tab:link_len_phantom_x}  % chktex 24
	\begin{center}
   	\begin{tabular}{|c||c|} \hline  % chktex 44
      Link Name & $[mm]$ \\ \hline  % chktex 44
      Coxa Link & 52  \\ \hline  % chktex 44
      Femur Link & 66  \\ \hline  % chktex 44
      Tibia Link & 130  \\ \hline  % chktex 44
    \end{tabular}
  \end{center}
\end{table}

\newpage

以上の式を用いて,PhantomXの脚の可動範囲を求めたものが\figref{fig:simu_leg_range}である.
\figref{fig:simu_leg_range}では,$\theta_c = 0$とした時の,x-z平面における脚の可動範囲を示している.
横軸をx軸,縦軸をz軸とし,単位は$[mm]$である.
黒色の実線で示される領域が実際の脚の可動範囲であり,緑色の実線で示される領域が近似された脚の可動範囲である.
赤色の点線で示される直線は遊脚の高さを表しており,この直線よりも高く上げることはない.
青色の線と頂点で示される図形は脚の概形である.

\figref{fig:simu_leg_range}より,$100 < x < 150$,$-50 < z < -20$の範囲は,
実際の脚の可動範囲と近似された脚の可動範囲が異なる領域になっているとわかる.
この領域内の点を脚の接地点として選択してしまうと,脚の可動範囲外に脚先が出てしまうため,
脚軌道を生成できず,脚が接地できないことによって転倒してしまう.
また,この領域の点を脚の接地点として選択していなくとも,
脚軌道がこの領域を通過する場合,脚の可動範囲外に脚先が出てしまうため,
矩形軌道の生成に失敗し,この領域を避けるような不完全な脚軌道が生成される.
プログラムは矩形軌道を描くことを前提にしているため,
このような不完全な脚軌道を生成すると,障害物に脚が引っかかってしまい歩行を継続できなくなる.
以上から,この領域が脚軌道生成の失敗の原因であると予想できる.

\figref{fig:simu_leg_range}は遊脚高さを$20 [mm]$,最小半径を$120 [mm]$とした時のものであるが,
遊脚高さを$25 [mm]$,最小半径を$130 [mm]$とした時の脚の可動範囲を\figref{fig:act_leg_range}に示す.
\figref{fig:simu_leg_range}の条件は波東らの研究\cite{Hato_Graph_search}のシミュレーション実験時の条件であり,
\figref{fig:act_leg_range}の条件は実機試験時の条件である.
\figref{fig:act_leg_range}では,$100 < x < 150$,$-50 < z < -20$の範囲の,
実際の脚の可動範囲と近似された脚の可動範囲が異なる領域の大きさが小さくなっていることがわかる.
実機試験時にあたり,条件を変更する必要があったことからも,
近似値と実際の値の差が脚軌道生成失敗に影響していると予想できるだろう.

\newpage

\begin{figure}[h]
  \begin{tabular}{cc}
      \begin{minipage}{0.5\textwidth}
          \centering
          \includegraphics[width=1.0\linewidth]{figure/chapter2/leg_range_120_20.png}
          \caption{Approximated Range of Motion in Simulation \newline}
          \label{fig:simu_leg_range} % chktex 24
      \end{minipage}
      \begin{minipage}{0.5\textwidth}
          \centering
          \includegraphics[width=1.0\linewidth]{figure/chapter2/leg_range_130_25.png}
          \caption{Approximated Range of Motion in Experimental Equipment}
          \label{fig:act_leg_range} % chktex 24
      \end{minipage}
  \end{tabular}
\end{figure}

% 予備実験の章
\section{歩行シミュレーションによる脚軌道生成失敗時の脚先位置の特定}

\subsection{シミュレーション実験の目的}
先行研究では脚軌道生成の失敗による動作の停止が報告された上,
その原因は脚先が脚の可動範囲の外を通ることによるものであると考察されてきた.
しかし,具体的にどのような理由で脚軌道生成が失敗するのかは明らかにされていなかった.
そのため予備実験として,波東らの研究\cite{Hato_Graph_search}の実機試験と同じ条件で歩行シミュレーション実験を行う.
実験の目的は脚軌道生成の失敗の回数,接地時の脚位置,そして脚軌道を確認することで,脚軌道生成失敗の原因を特定することとする.

波東らは段差のある地形と斜面のある地形でシミュレーション,および実機による直進歩行動作の実験を行った.
また,実機試験時にはシミュレーション実験と歩行時の条件を変更していた.
本シミュレーションでは,波東らの研究のシミュレーションと実機試験それぞれの条件で歩行シミュレーションを行う.

\subsection{シミュレーションの条件}
本研究室ではロボットの動作のシミュレーションを行うためのシミュレーターソフトウェアを自作し,
シミュレーション実験を行ってきた.
本論文でも同様に,シミュレーション実験に自作のシミュレーターソフトウェアを用いた.

シミュレーターはC++で記述されており,WindowsAPIを用いてGUIを実装し,ロボットを表示している.
また,GUIの表示のプログラムをより簡単に記述するため,
ゲームプログラミングに用いられるライブラリのDxLib\cite{Dxlib_Web}を用いている.
シミュレーターは物理演算を行っておらず,トルク不足や摩擦,脚先の滑りによるずれを考慮していない.
そのため,ロボットのアクチュエータは無限のトルクを持ち,脚先は滑りなく接地するものと仮定している.
本来これらのパラメータを考慮すべきではあるが,
本研究においては歩容パターン生成によって得られた脚接地点に脚先を届かせることが可能であるか確認することが目的であるため,
これらのパラメータは考慮しないこととしている.

\subsubsection{シミュレーションの計算環境}
シミュレーションの計算環境は\tableref{tab:simulation_env}に示した.

\subsubsection{モデルとするロボット}
モデルとするロボットは\figref{fig:phantomx_mk2}のPhantomX Mark I\hspace{-1.2pt}Iとする.

\subsubsection{歩行する地形}
歩行する地形を\figref{fig:terrain}に示す.
地形は5種類あり,それぞれ平地,上り段差,下り段差,上り斜面,下り斜面である.
実機試験の条件に合わせ,段差は上り,下りともに高さが$100 [mm]$とし,
斜面は上り,下りともに角度が$15 [\deg]$とした.

\subsubsection{歩行する時の条件}
シミュレーション時の歩行条件は以下の通りである.
\begin{itemize}
  \item 胴体姿勢は常に水平である.
  \item 最小半径を$120 [mm]$とする.
  \item 重心から見た遊脚高さを$-20 [mm]$とする.
  \item 胴体を地形から最小$30 [mm]$離す.
  \item 常に静的安定余裕は$10 [mm]$以上を保つ.
\end{itemize}

実機実験時の歩行条件は以下の通りである.
\begin{itemize}
  \item 胴体姿勢は常に水平である.
  \item 最小半径を$130 [mm]$とする.
  \item 重心から見た遊脚高さを$-25 [mm]$とする.
  \item 胴体を地形から最小$50 [mm]$離す.
  \item 常に静的安定余裕は$15 [mm]$以上を保つ.
\end{itemize}

以上の2つの条件で歩行シミュレーションを行う.

\subsubsection{シミュレーションの手順}
これらの地形で水平方向に$1200 [mm]$直進させ,
\figref{fig:terrain}のロボットの初期位置をランダムに変化させて計5回ずつ歩行させる.

このシミュレーションでは脚軌道生成に失敗した場合も,
仮に脚軌道生成に成功した場合と同様に歩行を継続させ,
脚軌道生成に失敗した場合の脚先の位置と回数を記録する.
これを各地形,各条件で行う.

% 計算環境の表
\begin{table}[htbp]
	\caption{Simulation Environment}
	\label{tab:simulation_env}  % chktex 24
	\begin{center}
   	\begin{tabular}{|c||c|} \hline  % chktex 44
      CPU & 11thGen Intel Core (TM) i5--11400  \\ \hline  % chktex 44
      RAM & 32.0GB  \\ \hline  % chktex 44
      OS & Windows 11 Home  \\ \hline  % chktex 44
      開発環境 & Visual Studio 2022 Community  \\ \hline  % chktex 44
      使用言語 & C++20  \\ \hline  % chktex 44
    \end{tabular}
  \end{center}
\end{table}

\newpage

% 地形の図
\begin{figure}[htbp]
  \begin{tabular}{cc}
    \begin{minipage}[t]{0.45\hsize}
      \begin{center}
      \includegraphics[width=1.0\linewidth]{figure/chapter2/map_flat.png}
      \text{(a) flat}
      \label{fig:flat_terrain} % chktex 24
      \end{center}
    \end{minipage} 
    &
    \begin{minipage}[t]{0.45\hsize}
      \begin{center}
      \includegraphics[width=1.0\linewidth]{figure/chapter2/map_100.png}
      \text{(b) up step}
      \label{fig:up_step_terrain} % chktex 24
      \end{center}  
    \end{minipage}
    \\
    &\\  % 空白を入れる
    \begin{minipage}[t]{0.45\hsize}
      \centering
      \includegraphics[width=1.0\linewidth]{figure/chapter2/map_-100.png}
      \centering
      \text{(c) down step}
      \label{fig:down_step_terrain} % chktex 24
    \end{minipage} 
    &
    \begin{minipage}[t]{0.45\hsize}
      \centering
      \includegraphics[width=1.0\linewidth]{figure/chapter2/map_15deg.png}
      \centering
      \text{(d) up slope}
      \label{fig:up_slope_terrain} % chktex 24
    \end{minipage}    
    \\
    &\\  % 空白を入れる
    \begin{minipage}[t]{0.45\hsize}
      \centering
      \includegraphics[width=1.0\linewidth]{figure/chapter2/map_-15deg.png}
      \centering
      \text{(e) down slope}
      \label{fig:down_slope_terrain} % chktex 24
    \end{minipage}     
    &
    \\
  \end{tabular}
  \caption{Terrain}
  \label{fig:terrain} % chktex 24
\end{figure}

\newpage

\subsection{シミュレーションの結果}
シミュレーション時の条件で歩行させたときの脚軌道生成失敗の回数を\tableref{tab:failure_count_simulation}に示す.
また,実機実験時の条件で歩行させたときの脚軌道生成失敗の回数を\tableref{tab:failure_count_experimental}に示す.
\tableref{tab:failure_count_simulation}では脚軌道生成の失敗率は多くの地形で$20 [\%]$を超えており,
動作を継続させることが難しいとわかる.
とくに脚の接地点が脚の可動範囲外になってしまうことが多く,
目標の地点まで脚先を届かせることができないため,転倒してしまう可能性が高い.
\tableref{tab:failure_count_experimental}は脚軌道生成の失敗率は多くの地形で$3 [\%]$程度であり,
\tableref{tab:failure_count_simulation}よりも著しく低いことがわかる.
また,脚の接地点が脚の可動範囲外になってしまうことはなく,
目標の地点まで脚先を届かせることができるため,転倒する可能性は低い.
そのため,脚が地形に引っかかってしまわない限り,動作を継続させることが可能であるといえる.


% 書き方 https://uec.medit.link/latex/table.html

\begin{table}[htbp]
  \caption{Failure Count of Simulation}
  \label{tab:failure_count_simulation}  % chktex 24
  \centering
  \begin{tabular}{|c|c|c|c|c|c|c|c|} \hline  % chktex 44
    \multirow{3}{*}{地形} & \multirow{3}{*}{グラフ探索} & \multicolumn{5}{c|}{失敗の回数} & \multirow{3}{*}{失敗率 $[\%]$} \\ \cline{3-7}  % chktex 44
     & & \multirow{2}{*}{脚接地点が} & \multicolumn{3}{c|}{脚軌道が可動範囲外を通る} & \multirow{2}{*}{総失敗} & \\ \cline{4-6}  % chktex 44
     & の回数 & 可動範囲外 & 遊脚時 & 接地時 & \begin{tabular}{c}胴体平行\\移動時\end{tabular} & 回数 & \\ \hline  % chktex 44
    平面     & 315 & 47 & 16 & 4 & 0 & 67 & 21.3 \\ \hline  % chktex 44
    上り斜面 & 460 & 63 & 11 & 22 & 1 & 97 & 21.1 \\ \hline  % chktex 44
    下り斜面 & 611 & 29 & 53 & 28 & 0 & 110 & 18.0 \\ \hline  % chktex 44
    上り段差 & 368 & 50 & 16 & 10 & 0 & 76 & 20.7 \\ \hline  % chktex 44
    下り段差 & 331 & 33 & 39 & 6 & 0 & 78 & 23.6 \\ \hline  % chktex 44
  \end{tabular}
\end{table}

\begin{table}[htbp]
  \caption{Failure Count of Experimental Equipment}
  \label{tab:failure_count_experimental}  % chktex 24
  \centering
  \begin{tabular}{|c|c|c|c|c|c|c|c|} \hline  % chktex 44
    \multirow{3}{*}{地形} & \multirow{3}{*}{グラフ探索} & \multicolumn{5}{c|}{失敗の回数}  & \multirow{3}{*}{失敗率 $[\%]$} \\ \cline{3-7}  % chktex 44
     & & \multirow{2}{*}{脚接地点が} & \multicolumn{3}{c|}{脚軌道が可動範囲外を通る} & \multirow{2}{*}{総失敗} & \\ \cline{4-6}  % chktex 44
     & の回数 & 可動範囲外 & 遊脚時 & 接地時 & \begin{tabular}{c}胴体平行\\移動時\end{tabular} & 回数 & \\ \hline  % chktex 44
    平面     & 351 & 0 & 9 & 2 & 0 & 11 & 3.13 \\ \hline  % chktex 44
    上り斜面 & 645 & 0 & 10 & 1 & 0 & 11 & 1.71 \\ \hline  % chktex 44
    下り斜面 & 867 & 0 & 20 & 9 & 0 & 29 & 3.34 \\ \hline  % chktex 44
    上り段差 & 461 & 0 & 6 & 10 & 0 & 16 & 3.47 \\ \hline  % chktex 44
    下り段差 & 383 & 0 & 9 & 0 & 0 & 9 & 2.35 \\ \hline  % chktex 44
  \end{tabular}
\end{table}

\newpage

シミュレーション時の条件で歩行させたとき,
脚軌道生成に失敗した脚先の位置を\figref{fig:result_of_simulation}に示す.
また,実機実験時の条件で歩行させたとき,
脚軌道生成に失敗した脚先の位置を\figref{fig:result_of_experimental}に示す.
どちらの図も\figref{fig:simu_leg_range}と\figref{fig:act_leg_range}の
$50 < x < 250$,$-200 < z < 0$の範囲を拡大したものである.
橙色の点は脚先が脚の可動範囲外になってしまった時の脚先の位置であり,
青色の点は脚軌道が脚の可動範囲外を通ってしまった時の脚先の位置である.
青色は始点,藍色は終点を表しており,明るい水色は中継点を表している.
それぞれ,すべての地形での結果をまとめて図示している.
この結果から,脚軌道生成に失敗した時の脚先の位置は,
実際の脚の可動範囲と近似された脚の可動範囲が異なる領域を通っていることがわかる.

\begin{figure}[h]
  \begin{tabular}{cc}
      \begin{minipage}{0.5\textwidth}
          \centering
          \includegraphics[width=1.0\linewidth]{figure/chapter2/result_simu.png}
          \caption{Result of Simulation}
          \label{fig:result_of_simulation} % chktex 24
      \end{minipage}
      \begin{minipage}{0.5\textwidth}
          \centering
          \includegraphics[width=1.0\linewidth]{figure/chapter2/result_exp.png}
          \caption{Result of Experimental Equipment}
          \label{fig:result_of_experimental} % chktex 24
      \end{minipage}
  \end{tabular}
\end{figure}

\subsection{脚軌道生成に失敗する原因の考察}
脚軌道生成に失敗した時の脚先の位置が,
実際の脚の可動範囲と近似された脚の可動範囲が異なる領域を通っていることから,
2.1.3章で予想した通り,実際の脚の可動範囲と近似された脚の可動範囲が異なる領域によって,
脚軌道生成に失敗することがわかった.

% 再評価手法の提案の章
\section{歩容パターンの再評価手法}
脚軌道生成の失敗を防ぐためには近似された脚の可動範囲を小さくするか,
より正確に近似することで誤差をなくせばよい.
しかし,前者は脚の接地点として選択される領域を縮めてしまうことになるため,
グラフ探索による手法の利点である効率的な動作ができなくなってしまうことや,
本来歩行可能であった地形を歩くことができなくなってしまうことなどの問題点がある.
また,後者は細かく正確に近似するとキャッシュしておく値が増えるため,
あらかじめ計算しておいた値にアクセスする際の呼び出しにかかる時間が長くなってしまう.
これにより,計算時間が長くなってしまう問題点がある.

先行研究では,シミュレーション時と実機試験時の条件を変更していたとおり,
前者の方法でこの問題に対応をしていた.
しかし\tableref{tab:walkable_terrain}の歩行条件と,
それを適用したときの歩行可能な地形を示した表からもわかるように,
上り段差と下り段差では歩行可能な地形が変わってしまい,
シミュレーションで歩くことができた地形を歩くことができなくなってしまった.
加えて,この上でも脚軌道生成に失敗してしまうことがあったため,
根本的にこの問題を解決するための方法を考える必要がある.

そこで,歩容パターングラフ探索の再評価を行うことで,脚軌道生成の失敗を防ぐ手法を提案する.
歩容パターングラフ探索の再評価とは,脚軌道生成に失敗した場合,
グラフ探索の処理をやり直して,脚軌道生成に失敗しないような歩容パターンを生成することと定義する.
グラフ探索の再評価の手順を\figref{fig:revaluation_methodolgy}に示す.
再評価手法では基本的には従来手法通りにグラフ探索による歩容パターン生成を行い,
脚軌道生成を行ってロボットを動作させる.
脚軌道生成に失敗した場合のみ,グラフ探索による歩容パターン生成をやり直して,
脚軌道生成に失敗しない新しい歩容パターンを生成する.
歩容パターン生成をやり直す際,脚軌道生成に失敗しないよう,脚の近似された可動範囲を狭め,
可動範囲外に接地することがないようにする.
こうして歩容パターングラフを再生成し,グラフ探索を再度行うことで,
脚軌道生成に失敗しないノードを取得することが可能となる.

再評価手法は前述した2つの解決法と比べ,それらのもつ問題点を克服している.
以下にそれぞれの解決策と比較した再評価手法の利点を示す.

\begin{description}
  \item[本来接地可能だった地点を選択できなくなってしまう問題]\mbox{}\\
    脚の可動域をある程度広く確保することが可能になり,
    可動範囲の境界に近い点を接地点として選択できるようになる.
    加えて,脚軌道生成に失敗する場合,狭められた可動域に切り替えて失敗を防ぐことができる.
    \\
  \item[値を呼び出す際にかかる時間が増えてしまう問題]\mbox{}\\
    再評価手法では歩容パターン生成をやり直す都合上,
    単純計算で探索にかかる時間が倍になってしまう.
    よってより正確に近似する方が実行時間的に優れると思える.
    しかし,歩容パターングラフの生成時には $10^5 \sim 10^6$ 程度の数のノードを生成するため,
    グラフ探索内の処理の時間の増加は,最終的な計算時間を大幅に増加させてしまう可能性がある.
    つまり再評価手法では計算時間の増加を約 2 倍程度で済ませることができるといえる.
\end{description}

\begin{table}[h]
	\caption{Walkable Terrain}
	\label{tab:walkable_terrain}  % chktex 24
	\begin{center}
   	\begin{tabular}{|c|c|c|} \hline  % chktex 44
      地形 & シミュレーション時 & 実機実験時 \\ \hline  % chktex 44
      上り段差 & $130 [mm]$ & $100 [mm]$ \\ \hline  % chktex 44
      下り段差 & $-130 [mm]$ & $-100 [mm]$\\ \hline  % chktex 44
      上り斜面 & $15 [\deg]$ & $15 [\deg]$ \\ \hline  % chktex 44
      下り斜面 & $-15 [\deg]$ & $-15 [\deg]$ \\ \hline  % chktex 44
    \end{tabular}
  \end{center}
\end{table}

% phantomx mk - 2 の図
\begin{figure}[h]
  \begin{center}
    \includegraphics[width=140mm, clip]{figure/chapter2/revaluation_method.png}
    \caption{Revaluation Methodology Procedures}
    \label{fig:revaluation_methodolgy} % chktex 24
  \end{center}
\end{figure}
                     %第2章
    %%%%%%%%%%%%%%%%%%%%%%%%%%%%%%%%%%%%%%%%%%%%%%%%%%%%%%%%%%%%%%%%%%%%%%%%
%%
%% 実験装置や開発機械.tex
%% LaTeX-2e 専用
%% 
%% 
%%        設計工学研究室 学位論文テンプレート
%%
%%                      作成日時    2010年 12月 17日
%%
%%%%%%%%%%%%%%%%%%%%%%%%%%%%%%%%%%%%%%%%%%%%%%%%%%%%%%%%%%%%%%%%%%%%%%%%

\chapter{実験装置や開発機械}\label{chapter:実験装置や開発機械}
第\ref{chapter:実験装置や開発機械}章では,~を述べる.

\section{全体機能とサブ機能の構造}
\section{○○機能の設計}
\section{××機能の開発}
\section{△△機能の実現}



            %第3章
    \input{text/experiment.tex}                                          %第4章
    \input{text/conclusion.tex}                                          %第5章
    %% 付録のtexファイルをまとめる

\appendix
% C++20への移行

\chapter{C++20への移行}\label{chapter:appendix_cpp20}
\section{C++20の新機能}
C++にはコンパイラの標準規格として,C++98,C++03,C++11などが存在する.
その中でもC++11以降は約3年に一度のペースで新しい規格が策定されている.
先行研究のプログラムでは,C++17を使用していたが,本研究ではC++20\cite{Thomas_C++20}を使用するように変更を行った.
C++20では,C++17からの変更点として,以下のようなものがある.
\begin{itemize}
  \item constexpr関数の制限緩和
  \item 標準ライブラリの多くの関数がconstexpr関数に変更
  \item conceptの導入
  \item std::number,std::formatの導入
\end{itemize}
これらの機能を使用することで,プログラムの最適化を行うことができる上,プログラムの可読性を向上させることができる.
グラフ探索の計算時間の短縮は,当研究室でグラフ探索の計算のアルゴリズムの最適化の研究が数々行われていることからもわかるように,
重要な課題である.
また,先行研究のプログラムでは,C言語とC++17の機能を混在して使用していたため,
可動性が低く,新しい機能を追加することが困難であった.
よって,C++20への移行は,プログラムの可動性を向上させるためにも重要である.
以下に各機能の詳細を記述する.

\section{constexpr変数・関数}
constexpr(コンストエクスプロー・コンストイーエックスピーアール)キーワードは,C++11から導入されたキーワードであり,
変数や関数に対して使用することができる.

constexpr変数は,コンパイル時に評価される変数であり,定数として扱うことができる.
C言語における同様の機能としては,マクロがあげられるだろう.
マクロはプリコンパイル時に文字を置換するため,定数として扱うことができる.
しかし,マクロは文字を置換して置き換えるため,型を持たない.
そのため,マクロを使用することで型の不一致や意図しないキャストの発生を引き起こす可能性がある.
また,マクロはグローバルスコープで定義されるため,名前の衝突を防ぐことができない.
constexpr変数は,マクロのようにふるまうことができるが,型を持つことやスコープを限定することができる.

constexpr関数はコンパイル時に評価される関数であり,
C言語におけるマクロ関数のような処理を実現するために使用される.
たとえば,\coderef{convert_func_as_macro}のようなプログラムを考える.

\begin{lstlisting}[caption=convert func as macro,label=convert_func_as_macro]
#include <iostream>

// マクロ関数として定義する.
#define CONVERT_TO_RAD(deg) (deg * 3.1415926535f / 180.0f)  

int main()
{
  float deg = 90.0f;
  
  // CONVERT_TO_RAD(deg)はプリコンパイル時に置換される.
  // よって deg * 3.1415926535f / 180.0f に置換される.
  std::cout << CONVERT_TO_RAD(deg) << std::endl;
}
\end{lstlisting}

% 1行開ける
\vskip \baselineskip  % chktex 1

このプログラムでは,マクロ関数を使用して,度数法で表された角度をラジアンに変換している.
プログラムを記述する際にはラジアンで角度を表現すると可読性が低くなるため,
度数法で記述することによる利点は大きいが,実際の処理ではラジアンで角度を表現する必要があるので,
このようなマクロが実際に使用されることは多いだろう.

しかし,マクロにはいくつかの問題点がある.
まず1つとして,マクロは常にグローバルスコープで定義されることである.
通常C++の開発においては,クラスや名前空間を使用して,変数や関数をスコープを限定して定義することが多い.
スコープを限定することは,変数や関数の名前が衝突することを防ぐことができるため,
大規模な開発や複数のライブラリを使用する場合には必須である.
だが,マクロは名前空間内に定義したとしてもグローバルスコープに展開されるため,
名前の衝突を防ぐことができないのである.

もう1つの問題点は,マクロは型の確認を行わないことである.
C++はpythonやjava scriptなどの動的型付け言語とは異なり静的型付け言語である.
そのため,コンパイル時に型の不一致や意図しないキャストを警告として出力することができ,
ランタイムエラーを防ぐことができる.
しかし,マクロはプリコンパイル時に文字を置換するだけであるため,型の確認を行わない.
そのためたとえば上記のマクロ関数において,引数をint型やdouble型,果てはその他のクラスなどに変更しても,
float型との掛け算演算子が定義されていればコンパイルは通ってしまう.
先行研究のプログラムではマクロ関数を使用している箇所が多く存在したため,
実際に浮動小数点型のfloat型とdouble型が混在している箇所が存在した.

これをconstexpr関数を使用することで,以下のように書き換えることができる.

\begin{lstlisting}[caption=convert func as constexpr,label=convert_func_as_constexpr]
#include <iostream>

// constexpr関数として定義する.
constexpr float ConvertToRad(float deg)
{
  return deg * 3.1415926535f / 180.0f;
}

int main()
{
  // deg を constexpr変数として定義する.
  constexpr float deg = 90.0f;
  
  // ConvertToRad(deg)はコンパイル時に実行される.
  // よって 1.57079632675f に置換される.
  std::cout << ConvertToRad(deg) << std::endl;
}
\end{lstlisting}

% 1行開ける
\vskip \baselineskip  % chktex 1

このようにconstexpr関数を使用することで,マクロ関数のようにコンパイル時に評価される関数を定義することができる.
また,constexpr関数はコンパイル時に評価が行われるため,コンパイル時に型の確認を行うことができる.
加えて,constexpr関数はスコープを限定することができるため,名前の衝突を防ぐことができる.

以上のようにマクロの持つ問題点を解決することができるが,constexpr関数の本当の利点はコンパイル時に処理が実行されることである.
\coderef{convert_func_as_constexpr} の14,15行目にあるように,引数を含めてコンパイル時に評価が可能であれば,
コンパイル時に関数が実行される.そのため,展開される内容は90.0f * 3.1415926535f / 180.0fではなく,
1.57079632675fである.
このようにコンパイル時に評価が行われることで,実行時に関数は実行されないため,
実行時の計算時間を短縮することができ,マクロに比べて高速な処理が可能である.
コンパイル時に評価される影響でコンパイル時間が長くなる問題点を除けば,
constexpr関数はマクロ関数に比べて多くの利点を持っているため,
使うことができる場面では積極的に使用することが望ましい.

constexpr関数内ではその性質上,実装にいくつかの制限がある.
しかしC++20では,以下のようにconstexpr関数の制限が緩和されている.
\begin{itemize}
  \item constexpr関数内でのtry-catchブロックを許可
  \item 定数式からの仮想関数の呼び出しを許可
  \item 可変サイズをもつコンテナのconstexpr化
\end{itemize}
これによって,標準ライブラリの多くの関数やクラスのメンバ関数がconstexpr関数に変更されている.
よく使用するものでは,std::vectorのコンストラクタやstd::stringのメンバ関数などがconstexpr関数に変更されている.

\section{Concept}
C++20では,concept(コンセプト)という新しい機能が導入された.
conceptは,テンプレートの型に対する制約を定義するための機能である.
C++17以前のテンプレートでは,型に対する制約を記述するために,SFINAE(Substitution Failure Is Not An Error)を使用していた.
SFINAEで型の制約を記述する場合,プログラムの内容に対して記述が煩雑になりがちである.
また,SFINAEはコンパイルエラーが発生した場合に,エラーメッセージがわかりにくい.
conceptを使用することで,SFINAEを使用することなく,型に対する制約を記述することができる.
以下に例を示す.

\begin{lstlisting}[caption=SFINAE,label=sfinae]
#include <iostream>
#include <type_traits>

template <typename T, 
typename = typename std::enable_if<std::is_integral<T>::value>::type>
func(T t)
{
  std::cout << t << std::endl;
}

int main()
{
  func(10);
  // func(10.0);  // コンパイルエラー
}
\end{lstlisting}

\begin{lstlisting}[caption=concept,label=concept]
#include <iostream>
#include <concepts>

template <typename T>
concept Integral = std::is_integral<T>::value;

template <Integral T>
void func(T t)
{
  std::cout << t << std::endl;
}

int main()
{
  func(10);
  // func(10.0);  // コンパイルエラー
}

\end{lstlisting}

\section{std::numbers,std::format}
C++20では,std::numbers,std::formatという新しい機能が導入された.
std::numbersはいくつかの定数を提供するための機能である.
たとえば,std::numbers::piは円周率を表す定数である.
同様な機能としてC言語にはM\_PIがあるが,C言語のM\_PIはマクロであるため,型を持たない.
そのため,M\_PIを使用する際には,型の不一致や意図しないキャストの発生を引き起こす可能性がある.
std::numbersはテンプレートを使用しているため,型を持つことができる.
任意の型を持つ定数は次のように使用することができる.

\begin{lstlisting}[caption=std::numbers,label=std_number]
#include <iostream>
#include <numbers>

int main()
{
  // std::numbers::pi_vは円周率を表す定数である.  
  std::cout << std::numbers::pi_v<float> << std::endl;  // float型
  std::cout << std::numbers::pi_v<double> << std::endl;  // double型
}

\end{lstlisting}

% 1行開ける
\vskip \baselineskip  % chktex 1

std::formatは,文字列をフォーマットするための機能である.
C言語では,printf関数を使用して文字列をフォーマットすることができるが,
printf関数はパラメーターの受け渡しをスタックで行う為,
printf内の置換指定子と不整合のあるパラメーターを指定するとプログラムの実行時にエラーが発生する.
std::formatはprintfとほぼ同じような機能を提供するが,パラメータが合わない場合は例外を投げる.
以下に例を示す.

\begin{lstlisting}[caption=std::format,label=std_format]
#include <iostream>
#include <format>

int main()
{
  // std::formatは文字列をフォーマットするための機能である.
  std::string str = std::format("Hello, {}!", "world");
  std::cout << str << std::endl;
}

\end{lstlisting}


% 脚の可動班を表示するプログラムの説明

\chapter{脚の可動域を表示するプログラム}\label{chapter:leg_range_python}

\section{概要}
近似的な脚の可動域が適切であるかを評価するためには,
PhantomXの可動域を正確に把握する必要がある.
そのためには可動域の可視化を行うことが必要だろう.
そこで,PhantomXの脚の可動域を可視化するプログラムを作成した.
プログラムは簡単のためPythonを用いて作成しており,
GitHubを通じてだれでも利用できるようにしている.
以下にプログラムの導入方法を説明し,
プログラムの仕様について述べる.

\section{導入方法}
プログラムはGitHubを通じて公開しているため,
まずはGitHubからプログラムをダウンロードする方法を説明する.
そしてPythonのプログラムをコンパイルする方法を説明する.

\subsection{Gitの使用方法}
Gitとは,ソースコードなどの変更履歴を記録・追跡するための分散型バージョン管理システムである.
分かりやすい例を挙げるならば,Googleドキュメントの履歴機能や,
Microsoft Wordの変更履歴機能のようなシステムをさまざまなプログラムのソースコードに対して適用したものといえるだろう.
また,GitHubとは,Gitを用いてソースコードを管理するためのサービスである.
GitHubを用いることで,ソースコードの変更履歴を保存することができるだけでなく,
自分のソースコードを公開したり,他の人が作成したソースコードをダウンロードしたりすることもできる.

Gitを使うため,まずはGitをインストールする必要がある.
次のリンクからGitのインストーラをダウンロードし,実行する.
インストーラではオプションの設定を変更する必要はないため,
すべてのページで変更をせずに「Next」を選択してインストールを行う.

\begin{itemize}
  \item Git for Windows https://git-scm.com/downloads (アクセス日 2023/12/30)
\end{itemize}

Gitは多くのアプリケーションとは違い,コマンドを用いて操作を行う.
そのため,インストールが完了しても,Gitのアプリケーションが起動することはない.
Gitのインストールが成功したかどうかは,コマンドを用いて確認する.
インストールが完了した場合,Gitのコマンドが使用できるようになっているはずである.
1度パソコンを再起動したのち,\figref{fig:git_bash}のように
スタートメニューに「Git Bash」と入力してGit Bashを起動する.
Git Bashが起動したら,\coderef{lst:git_version}のコマンドを入力する.
\figref{fig:git_version}のようにGitのバージョンが表示されれば,インストールは成功している.

\begin{lstlisting}[caption=Gitのバージョン確認コマンド, label=lst:git_version]
  git --version
\end{lstlisting}

\begin{figure}[h]
  \begin{tabular}{cc}
      \begin{minipage}{.5\textwidth}
          \centering
          \includegraphics[width=1.0\linewidth]{figure/appendix/git_bash.png}
          \caption{Git Bash}
          \label{fig:git_bash} % chktex 24
      \end{minipage}
      \begin{minipage}{.5\textwidth}
          \centering
          \includegraphics[width=1.0\linewidth]{figure/appendix/git_version.png}
          \caption{Display Git Version}
          \label{fig:git_version} % chktex 24
      \end{minipage}
  \end{tabular}
\end{figure}




\subsection{pythonのコンパイル方法}
本項目では,Windows 10におけるpythonのコンパイル方法を説明する.
Mac OSやLinuxにおけるpythonのコンパイル方法については,ここでは説明しない.
(しかし,「python Mac 環境構築」,「python Linux 環境構築」などで検索すると最適な方法が見つかるだろう.)

\section{プログラムの仕様}



\chapter{PhantomX Mark I\hspace{-1.2pt}Iのサーボモータ}\label{chapter:B}
\section{サーボモータの仕様}

PhantomX Mark I\hspace{-1.2pt}Iは計18個のサーボモータを搭載しており,
Dynamixel AX-12A,あるいはDynamixel AX-18Aを搭載したモデルがある.
今回使用したのはDynamixel AX-18Aを搭載したモデルであるため,
以下にDynamixel AX-18Aの仕様をまとめたWebページを掲載する.
(https://e-shop.robotis.co.jp/product.php?id=235 アクセス日 2023/9/15)

\begin{figure}
  \centering
  \begin{tabular}{cc}
    \fbox{\includegraphics[page=1, width=0.4\textwidth]{web_page/AX-18A_Manual.pdf}} &
    \fbox{\includegraphics[page=2, width=0.4\textwidth]{web_page/AX-18A_Manual.pdf}} \\
    Page 1 & Page 2 \\
  \end{tabular}

  \caption{Dscription AX-18A (page1--2)}
  \label{fig:ax-18_1-2}  % chktex 24
\end{figure}

\begin{figure}
  \centering
  \begin{tabular}{cc}
    \fbox{\includegraphics[page=3, width=0.4\textwidth]{web_page/AX-18A_Manual.pdf}} &
    \fbox{\includegraphics[page=4, width=0.4\textwidth]{web_page/AX-18A_Manual.pdf}} \\
    Page 3 & Page 4 \\
    \fbox{\includegraphics[page=5, width=0.4\textwidth]{web_page/AX-18A_Manual.pdf}} &
    \fbox{\includegraphics[page=6, width=0.4\textwidth]{web_page/AX-18A_Manual.pdf}} \\
    Page 5 & Page 6 \\
  \end{tabular}

  \caption{Dscription AX-18A (page3--6)}
  \label{fig:ax-18_3-6}  % chktex 24
\end{figure}

\begin{figure}
  \centering
  \begin{tabular}{cc}
    \fbox{\includegraphics[page=7, width=0.4\textwidth]{web_page/AX-18A_Manual.pdf}} &
    \fbox{\includegraphics[page=8, width=0.4\textwidth]{web_page/AX-18A_Manual.pdf}} \\
    Page 7 & Page 8 \\
    \fbox{\includegraphics[page=9, width=0.4\textwidth]{web_page/AX-18A_Manual.pdf}} &
    \fbox{\includegraphics[page=10, width=0.4\textwidth]{web_page/AX-18A_Manual.pdf}} \\
    Page 9 & Page 10 \\
  \end{tabular}

  \caption{Dscription AX-18A (page7--10)}
  \label{fig:ax-18_7-10}  % chktex 24
\end{figure}

\begin{figure}
  \centering
  \begin{tabular}{cc}
    \fbox{\includegraphics[page=11, width=0.4\textwidth]{web_page/AX-18A_Manual.pdf}} &
    \fbox{\includegraphics[page=12, width=0.4\textwidth]{web_page/AX-18A_Manual.pdf}} \\
    Page 11 & Page 12 \\
    \fbox{\includegraphics[page=13, width=0.4\textwidth]{web_page/AX-18A_Manual.pdf}} & 
    \\
    Page 13 &  \\
  \end{tabular}

  \caption{Dscription AX-18A (page11--13)}
  \label{fig:ax-18_11-13}  % chktex 24
\end{figure}

% 改ページする
\clearpage
 

                                          %付録(あれば)
    \backmatter{}                                                %章番号を付けない
    %%%%%%%%%%%%%%%%%%%%%%%%%%%%%%%%%%%%%%%%%%%%%%%%%%%%%%%%%%%%%%%%%%%%%%%%
%%
%% 謝辞.tex
%% LaTeX-2e 専用
%% 
%% 
%%        設計工学研究室 学位論文テンプレート
%%
%%                      作成日時    2010年 12月 17日
%%
%%%%%%%%%%%%%%%%%%%%%%%%%%%%%%%%%%%%%%%%%%%%%%%%%%%%%%%%%%%%%%%%%%%%%%%%

\chapter*{謝辞}\label{chapter:謝辞}
\addcontentsline{toc}{chapter}{謝辞}

本論文の研究と執筆にあたりその細部に至るまで終始懇切なる御指導と御鞭撻を賜りました,埼玉大学大学院理工学研究科 ○○○○教授に謹んで深謝の意を申し上げます.

本研究を共同遂行して頂いた,○○○○氏に御礼申し上げます.

本研究に懇切なる御助言を頂いた,○○○○氏に御礼申し上げます.

研究室において常に熱心な御討論を頂きました,OB・学生の方々に感謝の意を表します.

○○○○について有益なご助言を数多く賜りました○○○○氏(○○○○株式会社),に深謝申し上げます.
                                          %謝辞
    \begin{thebibliography}{99}

    \bibitem{Locomotion_for_difficult_terrain}
    Freyr Hardarson:
    ``Locomotion for difficult terrain'',
    1997.

    \bibitem{NEDO}
    国立研究開発法人 新エネルギー・産業技術総合開発機構:
    ``NEDO 先導研究プログラム 2021年度'',
    Vol.1,
    p.57, 
    2022. 

    \bibitem{Hirose_Static_stability_criterion}
    広瀬,塚越,米田: 
    ``不整地における歩行機械の静的安定性評価基準'', 
    J. of Robotic Systems,
    Vol.16, No.8, 
    pp.1076-1082, 
    1998.

    \bibitem{Prabir_Graph_search}
    Prabir K. Pal, K. Jayarajan: 
    ``Generation of Free Gait - A Graph Search Approach'',
    IEEE Transactions on Robotics and Automation,
    Vol.7, No.3,
    1991.

    \bibitem{Oki_Graph_search}
    大木,程嶋,琴坂: 
    ``多脚ロボットの不整地踏破を目標とするグラフ探索を用いた歩行パターン生成'', 
    ロボティクス・メカトロニクス講演会講演概要集,
    2015.   

    \bibitem{Nakaoka_Graph_search}
    中岡,程嶋,琴坂: 
    ``不整地における特定位置・脚着地点への遷移を目的とした多脚歩行ロボットの歩行動作計画'',
    日本機械学会関東支部総会講演会講演論文集,
    2016.

    \bibitem{Shina_Graph_search}
    椎名,程嶋,琴坂: 
    ``グラフ探索を用いた多脚ロボットの旋回歩容パターン生成'',
    日本機械学会関東支部総会講演会講演論文集,
    2018.

    \bibitem{Miura_Graph_search}
    三浦,程嶋,琴坂: 
    ``グラフ探索による多脚歩行ロボットの自由歩容パターン生成 第4報:出現頻度によるノード枝刈りを用いた探索時間の短縮'',
    日本機械学会関東支部総会講演会講演論文集,
    2019.

\end{thebibliography}
\endinput                                 %参考文献
\end{document}